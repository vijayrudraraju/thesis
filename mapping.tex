\resetdatestamp

\newcommand\Dfrac[2]{\frac{\displaystyle #1}{\displaystyle #2}}
\newcommand{\mathBF}[1]{\mbox{\boldmath $#1$}}
\newcommand{\C}[1]{\mathBF{#1}}

\chapter{Mapping}

\begin{comment}
\TeX{} does a marvelous job of setting mathematical formulas, most often
 choosing pleasing spacing.
However, on occasion one should intercede to improve the layout.
This chapter defines a few such occasions.
In addition, this chapter documents some features of the {\tt amsmath}
 package which overcome difficulties in typesetting some mathematical
 forms.
The {\tt amsmath} package is documented 
 in {\it The \LaTeX{} Companion} \cite{Goossens:1997}.

The modified setup is typeset as
\begin{equation}
  G(z) = \begin{cases}
           \Dfrac {P(z)}{1+z^{-1}} & \text{for $p$ even}, \\[1ex]
           P(z)                    & \text{for $p$ odd}.
         \end{cases}
\end{equation}

With the modified definitions, we get the following.
\def\hC#1{\C{\hat{#1}}\vphantom{\C{#1}}}           % hat vector
\def\htC#1{\C{\hat{\tilde{#1}}}\vphantom{\C{#1}}}  % hat, tilde vector
\def\tC#1{\C{\tilde{#1}}\vphantom{\C{#1}}}         % tilde vector
\begin{equation}
\begin{split}
  \C{d}^{(i)} &= \hC{v}^{(i)} - \htC{v}^{(i)} \\
  \C{n}^{(i)} &= \C{u}^{(i)} - \tC{v}^{(i)}
\end{split}
\end{equation}
\end{comment}

\section{Definition}

texture mapping
conceptual metaphor
function
morphism

The term \emph{mapping} is used in mathematics, computer science, and related technical fields to encapsulate the concept of a function and other related concepts. \emph{Mapping} is often shortened to \emph{map}. Like most technical terms used in multiple technical fields, the meaning of \emph{mapping} varies according to the context of the field that it is used. However in the case of mapping, the variation is subtle enough that a cursory understanding of the way mapping is used in a couple of contexts outside of DMIs might illuminate why it is a necessary and valuable concept to encapsulate in the context of new musical interfaces.

In mathematics terminology, the use of the term mapping varies subtly according to sub-discipline within mathematics and often between specific mathematicians, however it is acceptable to use the word as a synonym for \emph{function}. A function in mathematics describes an associative relationship between two sets of numbers. One set of numbers is referred to as a domain or set of inputs and the other set is referred to as a codomain, range, or set of outputs. Mathematicians often say that a function maps a domain to or onto a range, hence the reason for using the term \emph{mapping} as a synonym for \emph{function}.

Strictly speaking in mathematics, a function can only represent a \emph{one-to-one mapping}. This means that every element in the input (domain) set is associated by the function with one and only one element in the output (range) set. The corollary statement about a one-to-one mapping is that every element in the output set might be associated to one element, many elements, no elements, or all elements in the input set.

The following are examples of equations that represent one-to-one mappings:

\begin{equation} \label{eq1}
y = f(x) = x + 1
\end{equation}
\begin{equation} \label{eq2}
z = g(x) = x^{2}
\end{equation}

Common types of functions (one-to-one mappings) like \ref{eq1} and \ref{eq2} often use the set of all real numbers (any number that can be expressed in decimal notation) as both the input set and the output set. Since the set of all real numbers is infinitely large, the association from one input element to one output element is described by a mathematical transformation that when applied to any element of the input set (the set of all real numbers) produces the element of the output set (the set of all real numbers) that the input element is associated to or mapped onto. This is convenient because that means the function can be written very concisely in one line. 

\ref{eq1} represents a mapping labeled \begin{math}f\end{math} where the input element is labeled \begin{math}x\end{math} and the output element is labeled \begin{math}y\end{math}. \ref{eq2} represents a mapping labeled \begin{math}g\end{math} where the input element is labeled \begin{math}x\end{math} and the output element is labeled \begin{math}z\end{math}.

In computer science terminology, a mapping is an abstract data type or concrete data structure, commonly referred to as an \emph{associative array} or \emph{dictionary}.  

Abstractly and simply, the only way to store data of any kind (text, image, sound, signal) in a computer of any kind, such that one can retrieve the data later, is to put the data in a bucket that has a label. If the data has no label that is associated with it, then practically, it cannot be retrieved. 

In music information studies terminology, if one wants to be able to store a large amount of sheet music such that it can be found later, one needs a some system to catalog or index the material such that a piece of sheet music can be referred to by some label. In the case of a library, this label is often a call number, the title of the piece, or the name of the composer.

Just like a library can use many different labeling systems to catalog sheet music in a library so that it can be retrieved with minimum headaches, a computer system can utilize many different abstract data types or concrete data structures. In both cases, the best choice is dependent on the material or data being cataloged or indexed. A mapping in computer science parlance then, can be understood as a particular system for labeling data.

In particular, a mapping is a labeling system that uses a \emph{collection} of labels (keys) and a \emph{collection} of pieces of data (values), where every label in the first collection is associated with one or many of the pieces of data in the second collection. Then this structure encompassing the two collections and the list of associations between the two collections is given a higher level label that refers to the mapping as a whole.

This reframing of the term in the computer science context, as opposed to the mathematical context, clarifies one implication of the concept of mapping. One can specify multiple, unique mappings that each operate on the same collection of labels and the same pieces of data. Similarly, one can specify multiple, unique functions that each operate on the same input set and the same output set. 

So in a computer system, one can create multiple mappings that each associate the same collection of labels with the same collection of data differently. This characteristic of mappings is one of the primary reasons why mappings are a useful frame for understanding the potential of digital musical instruments. 

\section{Importance of Mapping in Musical Systems}

In an acoustic musical instrument, the DMI equivalents of the instrument's control mechanism and sound generation mechanism are \emph{intrinsically} coupled or bound to each other because the physical material of the instrument that forms the control mechanism also forms the sound generation mechanism. Making a decision that alters the control mechanism will significantly or subtly alter the sound generation mechanism.

In a DMI the control mechanism is composed of sensors. The physical gestures of the performer are not coupled to the sound generation through purely mechanical means, but through electrical means. Each sensor, whether it senses acceleration, orientation, touch, etc. produces an electrical signal that correlates to some physical measurement. This analog electrical signal is converted to a digital signal through some kind of analog to digital signal conversion and then this signal is processed by a computer and sent to the sound generation mechanism, which is a piece of software that produces audio samples.

A simple example makes clear that the control mechanism and sound generation mechanism are not \emph{intrinsically} coupled.

If one takes a control mechanism composed of touch sensors and replaces each of them with light sensors, then conditions the electrical signal emitted from the light sensors to fall within the same maximum and minimum range as the touch sensors were emitting, it is clear that the control mechanism has changed. But this change results in no change within the sound generation mechanism. The sound is generated by software that produces audio samples and thus is not effected by a change to the physical structure of the instrument.

In physical terms, when two formerly intrinsically coupled aspects of system are free to become uncoupled, a new degree of freedom is introduced.

In the context of DMI research this new degree of freedom is called a choice of \emph{mapping}.

\section{A Dedicated Mapping System: libmapper}

The creation, evolution, and stabilization of the design a digital musical instrument is inherently a tightly iterative process. A violin luthier must be comfortable enough playing a violin to develop an aesthetic sense of what affect the countless decisions that are made in the process of building a violin have on the experiential quality of the instrument that they are creating. Whether a particular decision is the right one can only truly be evaluated after the decision is made and someone plays the violin. 

Similarly, a team of people building a digital musical instrument will often make a construction decision, evaluate the effect of that decision, make some changes, and repeat. This is often referred to as iterative development and occurs whether or not it is every made explicit. 

The fact is that it is much easier to iterate and explore various designs of a virtual system then it is to explore designs of a physical system.

\section{Usability and Graphical User Interfaces}

\section{Sensor Networks}

\section{Digital Musical Instruments as Localized Sensor Networks}
