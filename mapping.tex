\resetdatestamp

\newcommand\Dfrac[2]{\frac{\displaystyle #1}{\displaystyle #2}}
\newcommand{\mathBF}[1]{\mbox{\boldmath $#1$}}
\newcommand{\C}[1]{\mathBF{#1}}

\chapter{Mapping}

\begin{comment}
\TeX{} does a marvelous job of setting mathematical formulas, most often
 choosing pleasing spacing.
However, on occasion one should intercede to improve the layout.
This chapter defines a few such occasions.
In addition, this chapter documents some features of the {\tt amsmath}
 package which overcome difficulties in typesetting some mathematical
 forms.
The {\tt amsmath} package is documented 
 in {\it The \LaTeX{} Companion} \cite{Goossens:1997}.

The modified setup is typeset as
\begin{equation}
  G(z) = \begin{cases}
           \Dfrac {P(z)}{1+z^{-1}} & \text{for $p$ even}, \\[1ex]
           P(z)                    & \text{for $p$ odd}.
         \end{cases}
\end{equation}

With the modified definitions, we get the following.
\def\hC#1{\C{\hat{#1}}\vphantom{\C{#1}}}           % hat vector
\def\htC#1{\C{\hat{\tilde{#1}}}\vphantom{\C{#1}}}  % hat, tilde vector
\def\tC#1{\C{\tilde{#1}}\vphantom{\C{#1}}}         % tilde vector
\begin{equation}
\begin{split}
  \C{d}^{(i)} &= \hC{v}^{(i)} - \htC{v}^{(i)} \\
  \C{n}^{(i)} &= \C{u}^{(i)} - \tC{v}^{(i)}
\end{split}
\end{equation}
\end{comment}

\section{Definition}

\section{Importance of Mapping in Musical Systems}

\section{A Dedicated Mapping System: libmapper}

\section{Usability and Graphical User Interfaces}

\section{Sensor Networks}

\section{Digital Musical Instruments as Localized Sensor Networks}