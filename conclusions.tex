\resetdatestamp

\chapter{Conclusions and Further Research}

This concludes the research presented in this thesis. The base impulse behind this whole endeavour is the belief that the interface of a system is as important, arguably more important, then the internals of the system. The internals of a system are tasked with maintaining the stability, reliability, and functionality of the system, but the interface of the system to the broader set of external systems, users, and world is ultimately what gives a system purpose. The history of computer science and electrical engineering is largely a history of ignoring the interface or relegating the interface to a mere curiosity, simply the interest of the types of individuals who are too ``artsy" to appreciate ``real" computer science and programming languages. This is true of not only graphical user interfaces, but the interfaces of server-based application programming interfaces, code libraries, object oriented classes, and even the documentation (I admit to this shortcoming) that allows the external world to understand the ``protocol" of the interface in plain, human language. The distinction between the \emph{system} and the \emph{interface} is, at its heart, an artifact of the history of computers and the specific individuals who devoted their lives to the endeavour.

Noticeably absent in this thesis is any mention of user testing or experimentation to see whether the Vizmapper interface actually accomplishes what is intended. Such testing would certainly be expected of a larger project on this topic and would be helpful to refine the interface in future iterations. However, to steal a phrase, I greet the conventional understanding of the utility of user testing with skepticism. Sometimes an interface should be molded to the user and other times the user should be molded by the interface. That is the nature of compelling non-computational interfaces; it stands to reason that the same would be true of computational interfaces. The type of long term study that would be necessary to observe how interfaces affect the mindset of a user is not something that is often (if at all) performed. However, I think that such a study would be far more compelling and useful than the typical user testing that is performed.

There are two concrete improvements that should be made to the current Vizmapper interface in the future. The first is to use Holten's connection bundling algorithm \cite{edgebundles2006} to reduce the visual clutter when many connections are displayed simultaneously. The second is to use some form of visual hashing to automatically generate a distinct visual pattern for each cluster and signal so that visually identical signals and clusters can be readily distinguished between, as one grows familiar with their particular Mapper network. There is much work to be done to further improve interfaces to Libmapper if the Mapper Tools project is ever to be used in a wide context.
