\resetdatestamp

\chapter{Vizmapper}

With the benefit of the research and thinking that is presented in the prior two chapters, it is now possible to embark on an informed analysis of the task and context at hand and to make informed design choices about the implementation of the system.

The graphical user interface to libmapper that is the result of these design choices is called Vizmapper.

The task of mapping is understood to be creating connections and mappings between a collection of signals being output by a collection of devices (usually with sensors) and a collection of inputs made available by various devices (usually with audio synthesis modules)  capable of receiving signals.

The context in which the task of mapping is performed is understood to be one to many programmers, engineers, composers, and/or musicians interested in experimenting with interactive musical systems for use in a creative (as opposed to business productivity or analysis) context.

\section{Task Analysis}

Task analysis is the primarily observational and heuristic analysis of the physical, mental, and contextual requirements for performing a specific task. As such, even in its most rigorous and quantitative forms, task analysis typically involves methods like discourse analysis, contextual inquiry, and video analysis. 
\begin{comment}
Task analysis and human-computer interaction: approaches, techniques, and levels of analysis - Abe Crystal, Beth Ellington
\end{comment}

The roots of scientific task analysis go back to 1911 when Frederick Taylor published \emph{The Principles of Scientific Management} \cite{crystal2004}. Taylor was interested in improving manufacturing productivity and incorporating understanding of human factors into work methods. Known commonly as Taylorism, he argued that managers should rigorously systematize the organization of workers based on empirical evidence. Of course, it is more accurate to refer to Taylor's discipline as something like job design; however, the relevant point here is that effort was being made to examine the efficacy of performing a task in one way as opposed to another. The psychological component of such job design was first examined by Harvard Business School between 1927 and 1932 at the Western Electric Hawthorne Plant. The studies essentially concluded that the psychology of individuals with the workplace contribute significantly to what workers produce and expect from their jobs.

It soon became commonplace for industrial engineers to incorporate analyses of production methods to improve interaction between humans and machines. As computers became a common machine that humans were interacting with in the workplace and the power and flexibility of computers as tools has expanded, human-computer interaction (HCI) is now a dedicated discipline. However, HCI, and thus user interface and data visualization design, has borne the fingerprints of a discipline that was ultimately derived from the standpoint of increasing productivity and improving job performance.

This may or may not present a problem, depending on whether we choose the characterize the use of computer in the context of accomplishing the task of mapping as an act of expression or an act of productivity. Atau Tanaka offers a line of thinking that is of use in resolving this problem \cite{tanaka2000}.

\begin{quote}
A tool can be improved to be more efficient, can take on new features to help in realizing its task, and can even take on other, new tasks not part of the original design specification. In the ideal case, a tool expands the limits of what it can do. It should be easy to use, and be accessible to [sic] wide range of naive users. Limitations or defaults are seen as aspects that can be improved upon.

A musical instrument's raison-d'etre, on the other hand, is not at all utilitarian. It is not meant to carry out a single defined task as a tool is. Instead, a musical instrument often changes context, withstanding changes of musical style played on it while maintaining its identity. A tool gets better as it attains perfection in realizing its tasks. The evolution of an instrument is less driven by practical concerns, and is motivated instead by the quality of sound the instrument produces. In this regard, it is not so necessary for an instrument to be perfect as much as it is important for it to display distinguishing characteristics, or "personality". What might be considered imperfections or limitations from the perspective of tool design often contribute to a "personality" of a musical instrument.

Computers are generalist machines with which tools are programmed. By itself, a computer is a tabula rasa, full of potential, but without specific inherent orientation. Software applications endow the computer with specific capabilities. It is with such a machine that we seek to create instruments with which we can establish a profound musical rapport.

The input device is the gateway through which the user accesses the computer software's functionality. As a generalist device, generalized input devices like the keyboard or mouse allow the manipulation of a variety of different software tools. Music software can be written to give musically specific capabilities to the computer. Input devices can be built to exploit the specific capabilities of this software. On this general platform, then, we begin to build a specialized system, each component becoming part of the total instrument description.
\end{quote}

This line of thinking suggests that the extent to which Vizmapper is a tool limits the extent to which the principles of user interface and data visualization design ought to play a role in the design process. HCI is much better suited to evaluating more objective notions like utility, tasks, and functionality than more subjective notions like personality, quality of sound, and musical rapport.

It is clear from Tanaka's definitions that a cursory analysis would conclude that Vizmapper serves as a tool than an instrument. However, the fact that Vizmapper is specifically a tool to accomplish the task of creating and modifying mappings \emph{within a musical instrument} suggests that the design must be treated more subtly in this particular context. As the purpose of Vizmapper is partly to make the connections between components in a musical instrument more malleable and more susceptible to experimentation for groups of non-programmers, the task bears some resemblance to a non-utilitarian task.



\section{System Design}

\section{Application of User Interface Design Principles}

\section{Application of Data Visualization Design Principles}
