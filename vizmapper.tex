\resetdatestamp

\chapter{Vizmapper}

With the benefit of the research and context that is presented in chapters 2 and 3, it is now possible to embark on an informed analysis of the mapping task for an MSN and to make informed design choices about the implementation of the system.

The graphical user interface to Libmapper that is the result of these design choices is called Vizmapper.

The task of mapping is understood from chapter 2 to be creating a set of associations and functional transformations between a set of signals being output by a set of devices (usually with sensors) and a set of inputs made available by various devices (usually with audio synthesis modules)  capable of receiving signals.

The context in which the task of mapping is performed is understood to be one to many programmers, engineers, composers, and/or musicians interested in experimenting with interactive musical systems for use in a creative (as opposed to business productivity or analysis) context.

\section{Task Analysis of DMI Mapping}
\begin{comment}
Task analysis and human-computer interaction: approaches, techniques, and levels of analysis - Abe Crystal, Beth Ellington
\end{comment}

The section on task analysis in chapter 3 concluded that as a technique it is best-suited to understanding how to design interfaces that maximize productivity as opposed to channeling creative expression. This may or may not present a problem, depending on whether we choose the characterize the use of computer in the context of accomplishing the task of mapping as an act of expression or an act of productivity. Atau Tanaka offers a line of thinking that applies to tools and instruments, but may be of use in attempting to resolve this problem \cite{tanaka2000}.

\begin{quote}
"A tool can be improved to be more efficient, can take on new features to help in realizing its task, and can even take on other, new tasks not part of the original design specification. In the ideal case, a tool expands the limits of what it can do. It should be easy to use, and be accessible to [sic] wide range of naive users. Limitations or defaults are seen as aspects that can be improved upon.

A musical instrument's raison-d'etre, on the other hand, is not at all utilitarian. It is not meant to carry out a single defined task as a tool is. Instead, a musical instrument often changes context, withstanding changes of musical style played on it while maintaining its identity. A tool gets better as it attains perfection in realizing its tasks. The evolution of an instrument is less driven by practical concerns, and is motivated instead by the quality of sound the instrument produces. In this regard, it is not so necessary for an instrument to be perfect as much as it is important for it to display distinguishing characteristics, or "personality". What might be considered imperfections or limitations from the perspective of tool design often contribute to a "personality" of a musical instrument.

Computers are generalist machines with which tools are programmed. By itself, a computer is a tabula rasa, full of potential, but without specific inherent orientation. Software applications endow the computer with specific capabilities. It is with such a machine that we seek to create instruments with which we can establish a profound musical rapport.

The input device is the gateway through which the user accesses the computer software's functionality. As a generalist device, generalized input devices like the keyboard or mouse allow the manipulation of a variety of different software tools. Music software can be written to give musically specific capabilities to the computer. Input devices can be built to exploit the specific capabilities of this software. On this general platform, then, we begin to build a specialized system, each component becoming part of the total instrument description." 
\end{quote}

This line of thinking suggests that the extent to which Vizmapper is not a tool limits the extent to which the principles of user interface and data visualization design ought to play a role in the design process. HCI is much better suited to evaluating more objective notions like utility, tasks, and functionality than more subjective notions like personality, quality of sound, and musical rapport.

It is clear from Tanaka's definitions that Vizmapper serves as a tool rather than an instrument. However, the fact that Vizmapper is specifically a tool for accomplishing the task of creating and modifying mappings \emph{within a musical instrument} suggests that the design must be treated more subtly in this particular context. As the purpose of Vizmapper is partly to make the connections between components in a musical instrument more malleable and more susceptible to experimentation for groups of non-programmers, the task bears some resemblance to a non-utilitarian task. It is reasonable to assume that if the evolution of an instrument is motivated by the quality of the sound that the instrument produces, then similarly the evolution of a mapping interface is motivated by the quality of the mappings and DMIs that the interface produces. This reality should be evaluating in parallel with the understanding that configuration of a mapping is a somewhat utilitarian task, either the interface allows a team to configure the mapping efficiently or it does not.

To see this, examine the task and to decompose the complex task of designing a musical instrument using an analysis process called \emph{hierarchical task analysis} \cite{annett1967}. 

Mark Marshall nicely distills the last decade of thinking into understanding the abstract components of a digital musical instruments by decomposing a digital musical instrument into 3 main components \cite{marshall2008}:

\begin{description}
\item \emph{The physical interface} containing the sensors, actuators, and physical body of the instrument.
\item \emph{The software synthesis system} which creates both the sonic output of the instrument and any visual, haptic and/or vibrotactile feedback.
\item \emph{The mapping system} in which connections are made between parameters of the physical interface and those of the synthesis system.
\end{description}

The design of these 3 components can be regarded as the 3 main subtasks of the overall task of designing a musical instrument. Each of these subtasks are themselves composed of more granular tasks, thus forming the beginnings of a hierarchical structure representing an analysis of the overall task.

\begin{description}
\item \emph{The physical interface}
\begin{description}
\item choose sensors that are capable of detecting the desired physical phenomena or gestures
\item choose actuators that are capable of inducing the desired physical phenomena or feedback
\item choose sensors that are made available as outputs for connections to the inputs of the synthesis system
\item choose actuators that are made available as inputs for connections from the outputs of the synthesis system
\item choose structural/aesthetic materials that combined with the sensors and actuators will form the composite form of the physical interface
\item design the shape of the composite physical interface
\end{description}
\item \emph{The software synthesis system}
\begin{description}
\item choose mechanism/algorithm for performing sound synthesis in software
\item choose mechanism/algorithm for generating visual, haptic, and/or vibrotactile feedback
\item choose parameters of sound synthesis that are made available as inputs for connections from the outputs of the physical interface and/or other components of the composite synthesis system
\item choose parameters of visual, haptic, and/or vibrotactile feedback synthesis that are made available as inputs for connections from the outputs of the physical interface and/or other components of the composite synthesis system
\item choose parameters of sound, visual, haptic, and/or vibrotactile synthesis that are made available as outputs for connections to the physical interface and/or other components of the synthesis system
\end{description}
\item \emph{The mapping system}
\begin{description}
	\item choose a collection of available outputs to start connections from
	\item choose a collection of available inputs to connect to specific outputs
	\item create mappings that specify how (if at all) to modify source signals from outputs to generate the desired destination signals for the inputs
\end{description}
\end{description}

This hierarchical model of musical instrument design generalizes well to anything from a self-contained handheld instrument to a massively distributed system spread over a large physical environment. It also works regardless of whether it is an individual or a team that is engaged in the instrument design. This decomposition also makes more clear why perhaps the mapping system is deserving of a dedicated system and user interface to manipulate. 

A good analogy is the choice of what collection of LEGO blocks to use as opposed to how to put the LEGO blocks together once the blocks are chosen. Procedurally, one chooses what types of blocks to use before deciding how to put the blocks together. Similarly, it makes sense to make certain choices about what pieces will be used for the physical interface and the software synthesis system before deciding how the mapping system will put the two collections of inputs/outputs together. It is difficult to map between two sets of signals when the sets of signals are not already defined. Of course, this model laying out the necessary choices to be made says nothing about \emph{what} choices should be made. That process is entirely dependent on the artistic intentions of the design team and the goal of a mapping tool is only to allow the choices that are made in the mapping subtask to be implemented as efficiently as possible.

\section{System Design}

As a piece of software, there are a limited number of options for implementing the system while ensuring that the system can be reliably used on the variety of systems that a team designing a musical system is likely to be using in the present and future. As of the time that this system has been implemented, the most common form factors for computers are desktops, laptops, tablets, and smartphones. Each form factor is typically loaded with one of a small collection of operating systems that are widely available and are easy to find expert knowledge about.

\begin{description}
\item[Desktops/Laptops] Windows (XP, Vista, 7), Mac OSX, Linux (Ubuntu, Debian, etc.)
\item[Tablets/Smartphones] Android, iOS
\end{description}

This is not an exhaustive list, however the point is that it is not practical (especially in a research context) to develop and maintain several versions of a software system that will work on a wide variety of operating systems and devices. However, the fact is that people will use whatever system they have available and the variety of devices that an artist or engineer is likely to use to interact with a sensor network is increasing. To design a practical system that is meant to work in a collaborative context, it is crucial to adapt to this reality.

Before beginning development on a software system, it is necessary to choose a development environment and a deployment environment.

There are a number of terms that are used in the remainder of the chapter that are helpful to define.

\begin{description}
\item[Development environment] the sofware environment that the developer uses to create code and executables
\item[Deployment environment] the software environment that the user uses to run executables
\item[Progammer Code] the human readable text that is written by the developer in a specific computer language and translated to produce an executable
\item[Machine code] the machine readable code that is often produced by a compiler and requires no further translation to be understood by a physical CPU
\item[Executable] an executable is the file that is produced by the developer and given to a user to run in their deployment environment - it is a file or collection of files that encompass the program that the user can \emph{run} without any further translation within their deployment environment - can be programmer code, machine code, or other forms of executable code  
\item[Machine] the physical CPU that understands and can directly execute machine code
\item[Compiler] takes programmer code as input and produces machine code as output
\item[Interpretor] takes programmer code as an executable to run a program
\end{description}

There are any number of very stable development environments that enable a software developer to write a single corpus of programmer code and translate into executables that can run on Windows, OSX, and Linux. This choices effects primarily the developer of the software and not the user of the software. The choice of deployment environment is much more consequential to the user because it effects how they will actually run the executable.

There are three common methods for a user to run software on their chosen machine.

The first method is for the user to run a pre-compiled machine code executable, which means that the developer has written code, translated the code with a compiler into machine-specific code, and produced a file that can be executed without any further translation on the user's specific machine/operating system combination. If there is no available pre-compiled binary for a specific machine architecture, the user can obtain the source code of the software and compile a binary file that works on their specific machine as long as a compiler exists for their machine.

The first method is an option when the software is developed in a \emph{compiled language} environment. There are also environments that use \emph{interpreted languages} or \emph{script languages}. An interpreted language differs from a compiled language because unlike a compiled language that uses the process of translating the program code into machine code with a compiler and producing a machine code executable to run the program, an interpreted language uses the process of interpreting the program code line by line directly when the program is run. The interpretor is responsible for interfacing with the machine for the interpreted code when the executable is run. In a interpreted language the programmer code and the executable are often the same file. A compiled generated executable interfaces directly with the machine because it has already been converted into machine code. 

A web application with internal logic is essentially an interpreted executable because a web browser that runs HTML, CSS, and Javascript downloads the programmer code and uses an internal interpretor to execute the program. It does not download a compiled executable. Distributing software as a script executable to be run in an interpretor is attractive because the burden for implementing the interpretor for various different machine architectures is moved to the developer of the programming language and the developer can count on a user being able to run their program as long as the interpretor for the language is available for the user's machine. The developer can write \emph{machine-independent} code.

Environments like Java use a third strategy and run software on \emph{virtual machines}. A virtual machine language is somewhat like a hybrid of a compiled language and interpreted language because a virtual machine language is compiled, but it is compiled to an intermediate machine code that is not the machine code of the actual physical machine that is running the program. It is compiled to the "machine code" of a virtual machine that must be implemented for all machines that one would like to run the program on. This is useful because the user can be given a single file that is not program code like a compiled language, but the developer still can count on the file being able to be run on any machine that has the virtual machine available on the system.

Given that the hope is for multiple team members to be running Vizmapper and working simultaneously on the mapping and that there is presumably a local network available to allow the various devices on the sensor network to communicate with each other, it makes sense for Vizmapper to be implemented as a web application. A web application is typically distributed as a two part software system where one part is called the \emph{client} and the other part is called the \emph{server}. One server can serve the web application to a relatively large number of clients. This is the source of another advantage of web applications. Client are typically run in a web browser and therefore can be reliably run on any machine that includes a modern web browser. Since Libmapper is compiled code, the server software can take care of interfacing with the library and communicating through the Mapper Protocol with the rest of the network. Because the server need only be run on one machine, this type of application is much easier to deploy on a collection of heterogenous machines.

Stephen Sinclair, the developer of Libmapper and a member of IDMIL, also developed a basic server written in Python (a popular interpreted language) that interfaces with Libmapper and also provides a convenient interface for a web browser client to execute Mapper commands by communicating with the server. This web server is called Webmapper.

\section{Application of User Interface Design Principles}

\section{Application of Data Visualization Design Principles}
