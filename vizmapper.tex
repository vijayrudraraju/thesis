\resetdatestamp

\chapter{Vizmapper}

With the benefit of the research and thinking that is presented in the prior two chapters, it is now possible to embark on an informed analysis of the task and context at hand and to make informed design choices about the implementation of the system.

The graphical user interface to Libmapper that is the result of these design choices is called Vizmapper.

The task of mapping is understood to be creating a set of associations and function transformations between a set of signals being output by a set of devices (usually with sensors) and a set of inputs made available by various devices (usually with audio synthesis modules)  capable of receiving signals.

The context in which the task of mapping is performed is understood to be one to many programmers, engineers, composers, and/or musicians interested in experimenting with interactive musical systems for use in a creative (as opposed to business productivity or analysis) context.

\section{Task Analysis of DMI Mapping}
\begin{comment}
Task analysis and human-computer interaction: approaches, techniques, and levels of analysis - Abe Crystal, Beth Ellington
\end{comment}

The prior section on task analysis concluded that as a technique it is best-suited to understanding how to design interfaces that maximize productivity as opposed to channeling creative expression. This may or may not present a problem, depending on whether we choose the characterize the use of computer in the context of accomplishing the task of mapping as an act of expression or an act of productivity. Atau Tanaka offers a line of thinking that is of use in attempting to resolve this problem \cite{tanaka2000}.

\begin{quote}
A tool can be improved to be more efficient, can take on new features to help in realizing its task, and can even take on other, new tasks not part of the original design specification. In the ideal case, a tool expands the limits of what it can do. It should be easy to use, and be accessible to [sic] wide range of naive users. Limitations or defaults are seen as aspects that can be improved upon.

A musical instrument's raison-d'etre, on the other hand, is not at all utilitarian. It is not meant to carry out a single defined task as a tool is. Instead, a musical instrument often changes context, withstanding changes of musical style played on it while maintaining its identity. A tool gets better as it attains perfection in realizing its tasks. The evolution of an instrument is less driven by practical concerns, and is motivated instead by the quality of sound the instrument produces. In this regard, it is not so necessary for an instrument to be perfect as much as it is important for it to display distinguishing characteristics, or "personality". What might be considered imperfections or limitations from the perspective of tool design often contribute to a "personality" of a musical instrument.

Computers are generalist machines with which tools are programmed. By itself, a computer is a tabula rasa, full of potential, but without specific inherent orientation. Software applications endow the computer with specific capabilities. It is with such a machine that we seek to create instruments with which we can establish a profound musical rapport.

The input device is the gateway through which the user accesses the computer software's functionality. As a generalist device, generalized input devices like the keyboard or mouse allow the manipulation of a variety of different software tools. Music software can be written to give musically specific capabilities to the computer. Input devices can be built to exploit the specific capabilities of this software. On this general platform, then, we begin to build a specialized system, each component becoming part of the total instrument description. 
\end{quote}

This line of thinking suggests that the extent to which Vizmapper is a tool limits the extent to which the principles of user interface and data visualization design ought to play a role in the design process. HCI is much better suited to evaluating more objective notions like utility, tasks, and functionality than more subjective notions like personality, quality of sound, and musical rapport.

It is clear from Tanaka's definitions that Vizmapper serves as a tool rather than an instrument. However, the fact that Vizmapper is specifically a tool for accomplishing the task of creating and modifying mappings \emph{within a musical instrument} suggests that the design must be treated more subtly in this particular context. As the purpose of Vizmapper is partly to make the connections between components in a musical instrument more malleable and more susceptible to experimentation for groups of non-programmers, the task bears some resemblance to a non-utilitarian task.

To see this, examine the task and to decompose the complex task of designing a musical instrument using a thought process advanced by \emph{Hierarchical Task Analysis} \cite{annett1967}. 

Mark Marshall nicely distills the last decade of thinking into understanding the abstract components of a digital musical instruments by decomposing a digital musical instrument into 3 main components \cite{marshall2008}:

\begin{description}
\item \emph{The physical interface} containing the sensors, actuators, and physical body of the instrument.
\item \emph{The software synthesis system} which creates both the sonic output of the instrument and any visual, haptic and/or vibrotactile feedback.
\item \emph{The mapping system} in which connections are made between parameters of the physical interface and those of the synthesis system.
\end{description}

The design of these 3 components can be regarded as the 3 main subtasks of the overall task of designing a musical instrument. Each of these subtasks are themselves composed of more granular tasks, thus forming the beginnings of a hierarchical structure representing an analysis of the overall task.

\begin{description}
\item \emph{The physical interface}
\begin{description}
\item choose sensors that are capable of detecting the desired physical phenomena or gestures
\item choose actuators that are capable of inducing the desired physical phenomena or feedback
\item choose sensors that are made available as outputs for connections to the inputs of the synthesis system
\item choose actuators that are made available as inputs for connections from the outputs of the synthesis system
\item choose structural/aesthetic materials that combined with the sensors and actuators will form the composite form of the physical interface
\item design the shape of the composite physical interface
\end{description}
\item \emph{The software synthesis system}
\begin{description}
\item choose mechanism/algorithm for performing sound synthesis in software
\item choose mechanism/algorithm for generating visual, haptic, and/or vibrotactile feedback
\item choose parameters of sound synthesis that are made available as inputs for connections from the outputs of the physical interface and/or other components of the composite synthesis system
\item choose parameters of visual, haptic, and/or vibrotactile feedback synthesis that are made available as inputs for connections from the outputs of the physical interface and/or other components of the composite synthesis system
\item choose parameters of sound, visual, haptic, and/or vibrotactile synthesis that are made available as outputs for connections to the physical interface and/or other components of the synthesis system
\end{description}
\item \emph{The mapping system}
\begin{description}
	\item choose a collection of available outputs to start connections from
	\item choose a collection of available inputs to connect to specific outputs
	\item create mappings that specify how (if at all) to modify source signals from outputs to generate the desired destination signals for the inputs
\end{description}
\end{description}

This model of musical instrument design generalizes well to anything from a self-contained handheld instrument to a massively distributed system spread over a large physical environment. It also works regardless of whether it is an individual or a team that is engaged in the instrument design. This decomposition also makes more clear why perhaps the mapping system is deserving of a dedicated system and user interface to manipulate. A good analogy is the choice of what collection of LEGO blocks to use as opposed to how to put the LEGO blocks together once the blocks are chosen. Procedurally, one chooses what types of blocks to use before deciding how to put the blocks together. Similarly, it makes sense to design most of the physical interface and the software synthesis system before deciding how the mapping system will put the two collections of inputs/outputs together.

\section{System Design}

As a piece of software, there are a limited number of options for implementing the system while ensuring that the system can be reliably used on the variety of systems that a team designing a musical system is likely to be using in the present and future. As of the time that this system has been implemented, the most common form factors for computers are desktops, laptops, tablets, and smartphones. Each form factor is typically loaded with one of a small collection of operating systems that are widely available and are easy to find expert knowledge about.

\begin{description}
\item \emph{Desktops/Laptops}
\begin{description}
\item Windows (XP, Vista, 7)
\item Mac OSX
\item Linux (Ubuntu, Debian, Gentoo)
\end{description}
\item \emph{Tablets/Smartphones}
\begin{description}
\item Android
\item iOS
\end{description}
\end{description}


This is not intended to be an exhaustive list, however the point is that it is not practical (especially in a research context) to develop and maintain several versions of a software system that will work on a wide variety of operating systems and devices. However, the fact is that people will use whatever system they have available and the variety of devices that an artist or engineer is likely to use to interact with a sensor network is increasing. To design a system that is meant to work in a collaborative context, it is crucial that this reality be adapted to.

Before beginning development on a piece of software one must choose a development environment and a deployment environment.

There are any number of very stable development environments that enable a software developer to write a single corpus of computer code and create applications that can run on Windows, OSX, and Linux.

There are three common ways for a user to run a piece of software on their chosen machine.

The user can install a pre-compiled binary image of the piece of software, which means that the developer has written code and produced a file that can be executed without any translation on the user's specific device/operating system combination.

The user can obtain the source code of the software and produce a file that works on their specific machine.

Virtual machines.

Scripts and interpreted environments. Included in this is a browser.

\section{Application of User Interface Design Principles}

\section{Application of Data Visualization Design Principles}
