\resetdatestamp

\chapter{User Interface and Data Visualization Design}

\section{Relevant History}



\section{Task Analysis}

Task analysis is a class of techniques used in HCI to guide the design of user interfaces. One of the central assumptions of user interface design and HCI is that different tasks require different interfaces. The corollary is that thinking carefully about the nature of the tasks that are meant to be accomplished by the interface will lead to better design decisions. Crystal and Ellington (2004) performed an in-depth comparative analysis of the dominant techniques and provide a good introduction to the motivations of task analysis:

\begin{quote}
Practitioners and researchers routinely advocate building user-centered systems which enable people to reach their goals, take account of natural human limitations, and generally are intuitive, efficient and pleasurable to use (Preece, Rogers and Sharp, 2002). Central to the design of such systems is a clear understanding of what users actually want to do: What are their tasks? What is the nature of those tasks? Many techniques have been proposed to help answer these questions. Task analysis techniques are particularly important because they enable rigorous, structured characterizations of user activity. They provide a framework for the investigation of existing practices to facilitate the design of complex systems.

Task analysis is especially valuable in the context of human-computer interaction (HCI). User interfaces must be specified at an extremely low level (e.g. in terms of particular interaction styles and widgets), while still mapping effectively to users� high-level tasks. Computer interfaces are often highly inflexible (when compared to interacting with a physical environment or another person). This inflexibility magnifies the impact of interface design problems, making the close integration of task structure and interface support especially crucial. \cite{crystal2004}
\end{quote}

Broadly, task analysis consists of the observational and heuristic analysis of the physical, mental, and contextual requirements for performing a specific task. As such, even in its most rigorous and quantitative forms, task analysis typically involves methods like discourse analysis, contextual inquiry, and video analysis. 

The roots of scientific task analysis go back to 1911 when Frederick Taylor published \emph{The Principles of Scientific Management} \cite{crystal2004}. Taylor was interested in improving manufacturing productivity and incorporating understanding of human factors into work methods. Known commonly as Taylorism, he argued that managers should rigorously systematize the organization of workers based on empirical evidence. Of course, it is more accurate to refer to Taylor's discipline as something like job design; however, the relevant point here is that effort was being made to examine the efficacy of performing a task in one way as opposed to another. The psychological component of such job design was first examined by Harvard Business School between 1927 and 1932 at the Western Electric Hawthorne Plant. The studies essentially concluded that the psychology of individuals with the workplace contribute significantly to what workers produce and expect from their jobs.

It soon became commonplace for industrial engineers to incorporate analyses of production methods to improve interaction between humans and machines. As computers became a common machine that humans were interacting with in the workplace and the power and flexibility of computers as tools has expanded, human-computer interaction (HCI) is now a dedicated discipline. 

This increased flexibility has meant that computers are becoming entangled in new areas of human behavior like music. Multiple techniques have developed to deal with this greater scope and complexity required of task analysis and each technique focuses on different aspects and contributes different insights into the nature of a human task.

The relevant point regarding the use of HCI techniques like task analysis to analyse artistic systems is that HCI, and thus user interface and data visualization design, has borne the fingerprints of a discipline that was ultimately derived from the standpoint of increasing productivity and improving job performance. 

\section{Graphical Perception and Information Seeking}
\begin{comment}
The Structure of the Information Visualization Design Space, Section 2 
controlled vs. automatic processing
connection

Visual Information Seeking: Tight Coupling of Dynamic Query Filters with Starfield Displays
\end{comment}
\section{Modal Interfaces}

\section{Filtering}
\begin{comment}
The Structure of the Information Visualization Design Space, Section 5
dynamic queries technique 

Visual Information Seeking: Tight Coupling of Dynamic Query Filters with Starfield Displays
\end{comment}
\section{Hierarchies and Taxonomies}
\begin{comment}
The Structure of the Information Visualization Design Space, Section 2
enclosure
\end{comment}
