\documentclass [12pt,letterpaper]{report}

% Standard packages
\usepackage{amsmath}		% Extra math definitions
\usepackage{graphics}		% PostScript figures
\usepackage{setspace}		% 1.5 spacing
\usepackage{longtable}          % Tables spanning pages
\usepackage{verbatim}

% Vijay packages
\usepackage{url}
\usepackage{graphicx}
\usepackage[table]{xcolor}
\usepackage[lofdepth,lotdepth]{subfig}

% Custom packages
\usepackage[first]{datestamp}	% Datestamp on first page of each chapter
\usepackage[fancyhdr]{McECEThesis}	% Thesis style
\usepackage{McGillLogo}		% McGill University crest

% $Id: ThesisEx.tex,v 1.1 2005/06/09 12:48:46 kabal Exp $

\usepackage{color}
\def\headrulehook{\color{red}}		% Color the header rule

%===== page layout
% Define the side margins for a right-side page
\insidemargin = 1.3in
\outsidemargin = 0.8in

% Above margin is space above the header
% Below margin is space below footer
\abovemargin = 1.1in
\belowmargin = 0.75in

%========= Document start

\begin {document}

%===== Title page

\title{A Tool for Configuring Mappings for Musical Systems using Wireless Sensor Networks}
\author{Vijay Rudraraju}
\date{\Month\ \number\year}
\organization{%
  \\[0.05in]
  \McGillCrest {!}{1in}\\	% McGill University crest
  \\[0.05in]
  Music Technology Area\\
  Schulich School of Music\\
  McGill University\\
  Montreal, Canada}
\note{%
  {\color{red} \hrule height 0.4ex}
  \vskip 2ex
  A thesis submitted to McGill University in partial fulfillment of the
  requirements for the degree of Master of Arts.
  \vskip 2ex
  \copyright\ \the\year\ Vijay Rudraraju
}

\maketitle

%===== Justification, spacing for the main text
\raggedbottom
\onehalfspacing
\pagenumbering{roman}

%===== Abstract, Sommaire & Acknowledgments
\section*{\centering Abstract}

Digital musical instruments, which are defined here as interactive musical systems containing a control mechanism and a sound generation mechanism, are powerful tools for analyzing performance practice and for transforming and reimagining the bounds of musical performance. However, the transitory nature of digital technology and the complexity of maintaining and configuring a digital musical instrument involving tens, if not hundreds, of interconnected, discrete components presents a unique problem. 

Even the most mechanically complex acoustic musical instruments, like a piano, are robust enough to withstand the daily grind without expert intervention by someone with intimate knowledge of the material and mechanical construction of the instrument. Furthermore, they are standardized enough that repairs can be conducted by any number of trained professionals. By contrast, digital musical instruments are often configured differently for each performance (this configurability being one of the virtues of a digital musical instrument), incorporate any number of non-standard pieces of hardware and software, and often can only be reliably configured by their creator.

This problem is exacerbated as the number of sensors that make up the control mechanism in an instrument increases and the interaction of the control mechanism with the sound generation mechanism grows more complex. This relationship between the control mechanism and the sound generation mechanism is referred to here as the "mapping" of the instrument. The mapping for an instrument represents the aspect of an instrument that is usually most configurable because it is defined by software (as opposed to hardware) and also most crucial to the character of the instrument. In the case of a digital musical instrument, being able to easily configure the musical instrument becomes a point of artistic freedom in addition to a point of maintainability.

This thesis builds upon work encompassed in two projects at the Input Devices and Musical Interaction Lab, the Digital Orchestra Project and Libmapper, to tackle the problem of building an interface/system for configuring a complex musical system without expert programming skills. The intent is to present a targeted survey of user interface design and data visualization design research through the years to inform the design of a graphical user interface for performing this configuration task.

\newpage

\section*{\centering Acknowledgments}

Many thanks to my supervisor, Professor Marcelo M. Wanderley, for his abundance of available time for his students and helping me integrate my personal work with the work of others at the Input Devices and Music Interaction Laboratory (IDMIL).

I am grateful to have been in the presence of all the students and professors of Music Technology at McGill University. Never have I been part of a group of people who were so eager to offer their time to others who needed help, regardless of whether they had something to gain from it. Special thanks to Joseph Malloch and Stephen Sinclair, without whose help and hours of work on the Mapper Tools project I would never have been able to even contemplate this research project.

I would never have made it to this point or survived the Montreal winters if it was not for Vincent Freour, a boundless source of energy who pulled me out into the sunlight even when I was determined to remain in my cave.

Finally, infinite gratitude to my parents Pandu and Padma who have always and will continue to support me, no matter how seemingly useless and lost I may appear through my many journeys.

%========== Tables of contents, figures, tables
\tableofcontents
\listoffigures
\listoftables

\newpage
\chapter*{List of Acronyms}\markright{List of Terms}

\begin{longtable}{ll}
  HCI		&	Human-Computer Interaction\\
  UID		&	User Interface Design\\
  DMI		&	Digital Musical Instrument\\
  EMI		&	Electronic Musical Instrument\\
  OSC	&	Open Sound Control\\
  IDMIL	&	Input Devices and Music Interaction Laboratory\\
\end{longtable}

\cleardoublepage
\pagenumbering{arabic}

%========== Chapters
\typeout{}
\resetdatestamp

\chapter{Introduction}

\begin{quote}
Most principles of design should be greeted with some skepticism, for word authority can dominate our vision, and we may come to see only through the lenses of word authority rather than with our own eyes.

What is to be sought in designs for the display of information is the clear portrayal of complexity. Not the complication of the simple; rather the task of the designer is to give visual access to the subtle and the difficult - that is,

the revelation of the complex. - Edward R. Tufte, 2001
\end{quote}

\section{Project Overview}

The purpose of this research project is to examine the effects of applying a thorough understanding of a problem domain and the a selection of theoretical principles in the areas of user interface design, data visualization, and human cognition to design a user interface that allows a user to solve problems in the domain as effectively as possible. 

\section{Layout}

\section{Contributions}


%==========
\typeout{}
\resetdatestamp

\newcommand\Dfrac[2]{\frac{\displaystyle #1}{\displaystyle #2}}
\newcommand{\mathBF}[1]{\mbox{\boldmath $#1$}}
\newcommand{\C}[1]{\mathBF{#1}}

\chapter{Mapping}

\begin{comment}
\TeX{} does a marvelous job of setting mathematical formulas, most often
 choosing pleasing spacing.
However, on occasion one should intercede to improve the layout.
This chapter defines a few such occasions.
In addition, this chapter documents some features of the {\tt amsmath}
 package which overcome difficulties in typesetting some mathematical
 forms.
The {\tt amsmath} package is documented 
 in {\it The \LaTeX{} Companion} \cite{Goossens:1997}.

The modified setup is typeset as
\begin{equation}
  G(z) = \begin{cases}
           \Dfrac {P(z)}{1+z^{-1}} & \text{for $p$ even}, \\[1ex]
           P(z)                    & \text{for $p$ odd}.
         \end{cases}
\end{equation}

With the modified definitions, we get the following.
\def\hC#1{\C{\hat{#1}}\vphantom{\C{#1}}}           % hat vector
\def\htC#1{\C{\hat{\tilde{#1}}}\vphantom{\C{#1}}}  % hat, tilde vector
\def\tC#1{\C{\tilde{#1}}\vphantom{\C{#1}}}         % tilde vector
\begin{equation}
\begin{split}
  \C{d}^{(i)} &= \hC{v}^{(i)} - \htC{v}^{(i)} \\
  \C{n}^{(i)} &= \C{u}^{(i)} - \tC{v}^{(i)}
\end{split}
\end{equation}
\end{comment}

\section{Definition}

\begin{comment}
texture mapping
conceptual metaphor
function
morphism
\end{comment}

The term \emph{mapping} is used in mathematics, computer science, and related technical fields to encapsulate the concept of a \emph{function} and other related concepts. \emph{Mapping} is often shortened to \emph{map}. Like most technical terms used in multiple technical disciplines, the meaning of \emph{mapping} varies according to the context of the field that it is used. However in the case of mapping, the variation is subtle enough that a cursory understanding of the way mapping is used in a couple of contexts outside of DMIs might illuminate why it is a necessary and valuable concept to encapsulate in the context of new musical interfaces.

\subsection{Mappings in Mathematics}

In the mathematics context, the use of the term mapping varies subtly according to sub-discipline within mathematics and often between specific mathematicians, however it is acceptable to use the word as a synonym for \emph{function}. A function in mathematics describes an associative relationship between two sets of numbers. One set of numbers is referred to as a domain or set of inputs and the other set is referred to as a codomain, range, or set of outputs. Mathematicians often say that a function "maps" a domain to or onto a range, hence the reason for using the term \emph{mapping} as a synonym for \emph{function} \cite{functionMapping}.

Strictly speaking in mathematics, a function can only represent a \emph{one-to-one mapping}. This means that every element in the input (domain) set is associated by the function with one and only one element in the output (range) set. A corollary statement about a one-to-one mapping is that every element in the output set might be associated to one element, many elements, no elements, or all elements in the input set.

The following are examples of equations that represent functions or one-to-one mappings:

\begin{equation} \label{eq1}
y = f(x) = x + 1
\end{equation}
\begin{equation} \label{eq2}
z = g(x) = x^{2}
\end{equation}

Basic types of functions like \ref{eq1} and \ref{eq2} often use the set of all real numbers (roughly, defined as any number that can be expressed in decimal notation) as both the input set and the output set. Since the set of all real numbers is infinitely large, the association from one input element to one output element is described by a mathematical transformation that when applied to any element of the input set (the set of all real numbers) produces the element of the output set (the set of all real numbers) that the input element is associated to or mapped onto. This is convenient because that means the function can be written very concisely in one line. 

\ref{eq1} represents a mapping/function labeled \begin{math}f\end{math} where the input element is labeled \begin{math}x\end{math} and the output element is labeled \begin{math}y\end{math}. \ref{eq2} represents a mapping/function labeled \begin{math}g\end{math} where the input element is labeled \begin{math}x\end{math} and the output element is labeled \begin{math}z\end{math}. 

To calculate the association that is implied by these two mathematical transformations, substitute any real number (since the set of all real numbers is the input set) for the symbol \begin{math}x\end{math}. By this logic, \ref{eq1} represents the set of associations where every real number \begin{math}x\end{math} is associated with the real number \begin{math}y\end{math} that is the value of \begin{math}x + 1\end{math}. \ref{eq1} represents the set of associations where every real number \begin{math}x\end{math} is associated with the real number \begin{math}z\end{math} that is the value of \begin{math}x^{2}\end{math}.

\subsection{Mappings in Computer Science}

In the computer science context, a mapping is an abstract data type or concrete data structure, commonly referred to as an \emph{associative array} or \emph{dictionary} \cite{assocarray2008}.  

Abstractly and simply, the one and only way to store data of any kind (the value "0", the value "1", some text, an image, a sound, a recorded signal) in a computer of any kind, such that one can retrieve the data later, is to put the data in some "bucket" that has some "label". The words "bucket" and "label" are not well-defined or widely used. The point is that if the data has no label that is associated with it, then practically, it cannot be retrieved. 

As an example from the music information retrieval context, if one wants to be able to store a large amount of sheet music such that it can be found later, one needs a some system to catalog or index the material such that a piece of sheet music can be referred to by some label. In the case of a library, this label is often a call number, the title of the piece, the name of the composer, etc.

Just like a reference library can use many different labeling systems to catalog the sheet music stored within its walls so that it can be retrieved with minimum headaches, a computer system can internally utilize many different abstract data types or concrete data structures. In both cases, the best choice is dependent on the material or data being cataloged or indexed. A mapping as a data structure in computer science parlance then, can be understood as a particular type of system for labeling data.

In particular, a mapping is a labeling system that is composed of a \emph{collection} of labels (keys) and a \emph{collection} of pieces of data (values), where every label in the first collection is associated with one or many of the pieces of data in the second collection. Then this structure encompassing the two collections and the list of associations between the two collections is given a higher order label that refers to the mapping as a whole. This allows one to construct a mapping of mappings, which lends itself nicely to hierarchical labeling systems.

This reframing of the term in the computer science context, as opposed to the mathematical context, clarifies one implication of the concept of mapping. One can specify multiple, unique mappings that each operate on the same collection of labels and the same collection of pieces of data. Similarly, one can specify multiple, unique functions that each operate on the same input set and the same output set. 

Note well, that although there is a distinct difference between a \emph{collection} and a \emph{set} in mathematical formalism, the colloquial understanding of the two terms, where they are interchangable, is sufficient for the purpose of this research and they will be used interchangably. 

So in a computer system, one can create multiple mappings that each associate the same collection of labels with the same collection of data differently. This characteristic of mappings is one of the primary reasons why mappings are a useful frame for understanding the potential of digital musical instruments. 

\section{Importance of Mapping in Digital Musical Instruments}

In an acoustic musical instrument, the equivalents of the "control mechanism" and "sound generation mechanism", as they are called in a DMI, are \emph{intrinsically} coupled or bound to each other because the physical material of the instrument that forms the control mechanism also forms the sound generation mechanism. Making a decision that alters the control mechanism will significantly or subtly alter the sound generation mechanism.

In a DMI, the control mechanism is composed of sensors. The physical gestures of the performer are not coupled to the sound generation through purely mechanical means, but through electrical means. Each sensor, whether it senses acceleration, orientation, touch, etc. produces an electrical signal that correlates to some physical measurement. This analog electrical signal is converted to a digital signal through some kind of analog to digital signal conversion and then this signal is processed by a computer and sent to the sound generation mechanism, which is a piece of software that produces audio samples.

A simple example makes clear that the control mechanism and sound generation mechanism are not \emph{intrinsically} coupled in a DMI.

If one takes a control mechanism composed of touch sensors and replaces each of them with light sensors, then conditions the electrical signal emitted from the light sensors to fall within the same maximum and minimum range as the touch sensors were emitting, it is clear that the control mechanism has changed. But this change results in no change within the sound generation mechanism. The sound is generated by software that produces audio samples and thus is not affected by a change to the physical structure of the instrument. It is true that changing the type of sensor might affect the dynamics of the analog signal produced and thus the dynamics of the sound produced, however whether or not this coupling occurs is dependent on the signal conditioning and digital conversion that is used and is no longer an intrinsic aspect of the physical structure.

In physical terms, when two formerly intrinsically coupled aspects of system are free to become uncoupled, a new degree of freedom is introduced.

In the context of DMI research this new degree of freedom is called a choice of \emph{mapping}. Mapping as used in this context is very similar to how the term is used in the mathematics and computer science context. The set of sensors embedded in the control mechanism produces a set of signals. The sound generation mechanism produces a specific sound depending on a set of signals dictated by the specifics of the software that composes the sound generation. Therefore, for any given control mechanism and sound generation mechanism there is a choice of how to associate the two sets of signals with each other. Importantly, this is a choice that is not independently available when constructing, for example, a violin because the two sets of signals (if this distinction between control mechanism and sound generation mechanism can be imagined in a violin) in that case are physically coupled and inseparable. It is not a choice that is explicitly made when constructing an acoustic instrument.

This physical decoupling between the control mechanism, the mapping system, and the sound generation mechanism creates a modularity between the three components that can be exploited to simplify the DMI development and maintenance process.

\section{A Dedicated Mapping System: libmapper}

The creation, evolution, and stabilization of the design a digital musical instrument is inherently a tightly iterative process. A violin luthier must be comfortable enough playing a violin to develop an aesthetic sense of what effect the countless decisions that are made in the process of building a violin have on the experiential quality of the instrument that they are creating. Whether a particular decision is the right one can only truly be evaluated after the decision is made and someone plays the violin. 

Similarly, a team of people building a DMI will often make a construction decision, evaluate the effect of that decision, make some changes, and repeat. This is often referred to as iterative development and occurs in any serious design process whether or not it is ever made explicit \cite{iterative2003}. 

Because such iterative development is so beneficial to the development of DMIs, people often use various platforms and frameworks as common building blocks for such systems. Common platforms allow one too quickly test ideas for a DMI and focus on experimenting with the aspects of the control mechanism and sound generation that are unique to the particular project before investing too much time and money into custom parts without assurance that the ideas are even viable in a general sense. For the control hardware there are platforms like Arduino. Coding environments like Max/MSP and code libraries like STK simplify the development of synthesizer modules for sound generation. In an effort to provide a similar foundation for creating mapping systems, the Digital Orchestra Toolbox was created as part of the McGill Digital Orchestra project and includes components representing the common subroutines used in a mapping system. 

More recently, this functionality has been reimplemented as a C library called libmapper at the IDMIL at McGill University. Libmapper has several characteristics which make it a desirable system for configuring mappings. 

One is that a mapping between control and sound generation can be modified without recompiling any code, restarting any system, or reloading any script. Typically, if a mapping is specified in software through a Max/MSP program, C program, etc. the part of the code that specifies the mapping must be modified and recompiled or reinterpreted before the new mapping becomes active. Any DMI that embeds libmapper in its control and sound generation software can have its mapping modified simply by being sent specific messages as specified by the mapping protocol. 

Another useful characteristic is that in the case of a local network with many different control mechanisms and sound generation mechanisms available, no central device is need to facilitate communication between devices. If one component on the network expriences some difficulties, the other devices will still be able to keep the mappings between the remaining devices operating.  

Lastly, and of particular relevance to the primary topic of this paper, is that the current state of mappings between various devices on the network can be viewed and modified by any graphical user interface that embeds Libmapper. In Libmapper parlance, these applications can register as \emph{monitors}. Because of this capability, multiple members of a team can be viewing and modifying the same mappings for multiple DMIs using different computers and using different graphical user interfaces.  

Seen from a high conceptual level, libmapper is an implementation of a communication protocol that was created for mappings. 

A communications protocol is a rigorous and formal description of a message format and rules for exchanging messages adhering to the protocol. The term protocol usually refers to the formal description of the protocol and is distinct from the implementation of the protocol. A protocol without an implementation is metaphorically like a constitution without a government. The mapping protocol is itself defined in terms of the Open Sound Control (OSC) protocol. This is a common practice for constructing higher level protocols. The Internet is essentially a particular stack of protocols that are layered on top of each other and specify how devices and routers are to be addressed, open connections with each other, and interpret messages.

In the case of Libmapper, the protocol layer that OSC itself is implemented on top of is called the User Datagram Protocol (UDP). The other common protocol that can be used at this layer in the Internet protocol stack is the Transmission Control Protocol (TCP), which is the protocol that the well-known Hypertext Transfer Protocol (HTTP) is layered on top of. A web browser implements HTTP so that it can fetch and display web pages from remote systems. As a protocol, OSC does not assume a particular Transport (the name given to the layer in the protocol that UDP and TCP occupy - HTTP occupies the layer on top of the Transport layer, known as an Application layer) protocol that it is to be implemented as. This is different from HTTP, which is specifically a TCP-based protocol.

Libmapper then allows one to configure a mapping relationship between any collection of systems (control systems, sound generation systems, and anything else) by embedding Libmapper into the code for each system, thus allowing any system on a local network to easily implement the mapping protocol without in depth knowledge of the protocol.

The implementation

\section{Usability and Graphical User Interfaces}

\section{Sensor Networks}

\section{Digital Musical Instruments as Localized Sensor Networks}


%==========
\typeout{}
\resetdatestamp

\chapter{User Interface and Data Visualization Design}

\section{Relevant History of Human-Computer Interfaces}

The concerns of user interface designers have changed considerably since people first started interacting with programmable digital computers for processing, storing, and retrieving information. 

\subsection{Punch cards}

The first interfaces often depended on punch cards and paper tape as control mechanisms and line printers for output and program feedback. On machines used during the 1950s and 1960s, in order to input a program and a dataset into a computer, first one prepares a deck of punched cards using a typewriter-like machine, then one feeds the deck to the computer \cite{oshistory2011}. Optionally on some machines, one can mount reels of magnetic tape that store oft-used or previously generated datasets or compiled software. On these early machines, a particular program being run from a deck of punch cards would occupy the resources of the entire computer. Any hardware devices required by a particular program like punch card readers and line printers had to be handled completely by the program. 

\emph{Operating systems} development grew out of the growing complexity of computer systems involving many cooperating devices and the mirrored growth of the complexity of the programs that were written for them. It makes little sense for a user of a machine to write basic operating software for interfacing with standardized perhipheral devices everytime they want to run a program on such a machine. Especially, since any user of a computer during the 1960s likely needed to enter a queue to reserve time to run their program on a single shared machine and any time wasted debugging low-level interactions with input and output devices was surely frusterating. Predictably, machines begain appearing with libraries of code that took care of low-level control and provided easier access to input and output devices as always resident services. This collection of software that is permenantly resident on a machine became known as a \emph{batch monitor} \cite{os2000}. A user's program could link to these preloaded libraries without including the operating logic explicitly within their own program code. This shift is often cited as the genesis of the modern operating system.

In addition to providing access to input/output devices, the monitor often provided services to perform error checking on user submitted programs and generating useful feedback to the user concerning the progress of execution of a user program. The idea of generating "useful" feedback, and what that might mean, can be understood as the first instance of an explicitly designed user interface \cite{unix2008}.

\subsection{Command-line}

Command-line interfaces are the next step in the evolution of computer interfaces and the link between batch systems of the 1950s and what would be recognized as a \emph{graphical user interface} (GUI) today. With continual, steady reduction of the amount of time needed to complete a computation cycle, it became possible to interact with the computer with a series of requests and responses expressed as specially formatted strings of text using a specialized vocabulary. Requests could be completed in seconds, no longer hours and days, and it made sense for the user to simply wait for the request to be completed before entering in the request. As opposed to queueing up a mult-stage program and waiting for all the stages to be completed, the user can change their mind about the structure of later stages in the program in response to feedback from earlier stages. This introduces the possibility for software to explore a set of possibilities with the guidance of the user and allows for a type of interactivity not possible in the 1950s. 

Although the earliest command-line systems borrowed typewriter-like teletypes (used for telegraph transmission) as the input and output mechanism, by the 1970s video display terminals were used with computers for providing text feedback on a virtual canvas of pixels that could be rapidly and reversibly modified (unlike teletype printers) and it became possible for a program to display an interface that could be called visual as well as textual \cite{unix2008}. This allowed programmers to create the first computer games and text editors that relied on this capability of video display terminals.

\subsection{Graphical}

Further reduction of the amount of time needed to execute individual operations within a program resulted in the ability for the computer to communicate with multiple input/output devices in realtime. Typical devices that a modern computer program expects to have access to through the operating system include a color monitor wherein each pixel in the monitor had a separate referenceable address, a graphics card to help with processing of 2d/3d graphics operations, a mouse, a keyboard, and a sound card connected to audio speakers.

Much of the common grammar of all the popular graphical user interfaces in use today were developed by two particular projects. The first is called NLS/Augment (NLS stands for oN-Line System) and was designed by Douglas Engelbart and his team at the Stanford Research Institute. During a famous, public demonstration of the system in 1968 (affectionately, often referred to as "The Mother of All Demos"), Engelbart proceeded to demonstrate the use of a computer mouse, a graphical display with multiple windows, and hyperlinks among many other notable advancements in human-computer interaction. 

The second groundbreaking project, the Xerox Alto, came from the Xerox Palo Alto Research Center (PARC) in 1973. It is perhaps the first computer designed from inception to be dedicated to the use of a single person. Although the monitor displayed only black and white pixels, the graphical user interface of the operating system contained buttons, windows, scrollbars, sliders, and many of the logical GUI components that are standard components of any program with a GUI.

It is primarily the ideas that these two projects consolidated and introduced to the wider world that eventually became the impetus for the recognition of \emph{Human-Computer Interaction} and, of more precise relevance to this paper, \emph{User Interface Design} as important areas of study of significant relevance to the broader computer science community. 

\section{Task Analysis}

Task analysis is a class of techniques used in HCI to guide the design of user interfaces. One of the central assumptions of user interface design and HCI is that different tasks require different interfaces. The corollary is that thinking carefully about the nature of the tasks that are meant to be accomplished by the interface will lead to better design decisions. Crystal and Ellington (2004) performed an in-depth comparative analysis of the dominant techniques and provide a good introduction to the motivations of task analysis:

\begin{quote}
Practitioners and researchers routinely advocate building user-centered systems which enable people to reach their goals, take account of natural human limitations, and generally are intuitive, efficient and pleasurable to use (Preece, Rogers and Sharp, 2002). Central to the design of such systems is a clear understanding of what users actually want to do: What are their tasks? What is the nature of those tasks? Many techniques have been proposed to help answer these questions. Task analysis techniques are particularly important because they enable rigorous, structured characterizations of user activity. They provide a framework for the investigation of existing practices to facilitate the design of complex systems.

Task analysis is especially valuable in the context of human-computer interaction (HCI). User interfaces must be specified at an extremely low level (e.g. in terms of particular interaction styles and widgets), while still mapping effectively to users� high-level tasks. Computer interfaces are often highly inflexible (when compared to interacting with a physical environment or another person). This inflexibility magnifies the impact of interface design problems, making the close integration of task structure and interface support especially crucial. \cite{crystal2004}
\end{quote}

Broadly, task analysis consists of the observational and heuristic analysis of the physical, mental, and contextual requirements for performing a specific task. As such, even in its most rigorous and quantitative forms, task analysis typically involves methods like discourse analysis, contextual inquiry, and video analysis. 

The roots of scientific task analysis go back to 1911 when Frederick Taylor published \emph{The Principles of Scientific Management} \cite{crystal2004}. Taylor was interested in improving manufacturing productivity and incorporating understanding of human factors into work methods. Known commonly as Taylorism, he argued that managers should rigorously systematize the organization of workers based on empirical evidence. Of course, it is more accurate to refer to Taylor's discipline as something like job design; however, the relevant point here is that effort was being made to examine the efficacy of performing a task in one way as opposed to another. The psychological component of such job design was first examined by Harvard Business School between 1927 and 1932 at the Western Electric Hawthorne Plant. The studies essentially concluded that the psychology of individuals with the workplace contribute significantly to what workers produce and expect from their jobs.

It soon became commonplace for industrial engineers to incorporate analyses of production methods to improve interaction between humans and machines. As computers became a common machine that humans were interacting with in the workplace and the power and flexibility of computers as tools has expanded, human-computer interaction (HCI) is now a dedicated discipline. 

This increased flexibility has meant that computers are becoming entangled in new areas of human behavior like music. Multiple techniques have developed to deal with this greater scope and complexity required of task analysis and each technique focuses on different aspects and contributes different insights into the nature of a human task.

The relevant point regarding the use of HCI techniques like task analysis to analyse artistic systems is that HCI, and thus user interface and data visualization design, has borne the fingerprints of a discipline that was ultimately derived from the standpoint of increasing productivity and improving job performance. 

\section{Graphical Perception and Information Seeking}

\begin{quote}
Graphics is the visual means of resolving logical problems. \cite{bertin1981}
\end{quote}

This idea from Bertin, who conducted one of the first attempts to provide a theoretical foundation to information visualization, summarizes the hope for an alternative interface for configuring a Libmapper network. 

The task of mapping is a logical problem, as well as an artistic one. It is a logical problem in the sense that one cannot make connections haphazardly between any pair of input and output signals and expect the network mapping that is produced to be interesting as a DMI. One must still infer through some form of reasoning the suitability of a particular connection in the context of the particular devices on the network, the nature of the signals that the devices produce, and the broader artistic intentions of the design team. 

It could be that one output signal generates floating-point decimal values and one input signal accepts only integer values. Or if a specific output signal tends to generate an essentially static signal regardless of the gestures performed with the MSN, it would make for a very dull and non-dynamic performance to connect this signal to the pitch input signal of a audio synthesizer. Of course, as previously stated one does not typically make use of metrics like productivity or performance in an artistic context, so the logical problem is different. But even an artistic project has goals and intentions, though they are considerably harder to define in words. Visualizing the Libmapper network effectively within a user interface will likely help a team of people (as long as there is agreement about the artistic and other intentions of the scenario that the MSN is to be used in), resolve these logical problems involving network topology and signal transformation more effectively. 

According to Bertin, graphics have at least two distinct uses. The first is as a means of communicating some information. The second is as a medium for graphical processing, defined as using the manipulation and perception of graphical objects to understand the information \cite{card1997}. Any graphical interface that is monitoring a Libmapper network will likely incorporate both uses of graphics. The current state of the network, including the namespaces and properties of all registered devices and signals and all signal connections and transformations, needs to be able to be gleaned from the interface. Also, the interface must allow the user to manipulate the graphics in such a way as to facilitate the connection and disconnection of signals with or without functional transformations applied to the signals.

\begin{comment}
The Structure of the Information Visualization Design Space, Section 2 
controlled vs. automatic processing
connection

Visual Information Seeking: Tight Coupling of Dynamic Query Filters with Starfield Displays
\end{comment}
\section{Recognition vs. Recall}
\section{Modal Interfaces}

\section{Filtering}
\begin{comment}
The Structure of the Information Visualization Design Space, Section 5
dynamic queries technique 

Visual Information Seeking: Tight Coupling of Dynamic Query Filters with Starfield Displays
\end{comment}
\section{Hierarchies and Taxonomies}
\begin{comment}
Magic number seven, George Miller
The Structure of the Information Visualization Design Space, Section 2
enclosure
\end{comment}


%==========
\typeout{}
\resetdatestamp

\chapter{Vizmapper}

With the benefit of the research and context that is presented in Chapters 2 and 3, it is now possible to embark on an informed analysis of the mapping task for an MSN and to make informed decisions about the implementation of the system.

The graphical user interface to Libmapper that is the result of these design choices is called Vizmapper.

The task of mapping is understood, from Chapter 2, to be creating a set of associations and functional transformations between a set of signals being output by a set of devices (usually with sensors) and a set of inputs made available by various devices (usually with audio synthesis modules)  capable of receiving signals.

The context for the mapping task is understood to be one where many programmers, engineers, composers, and/or musicians are interested in experimenting with interactive musical systems for use in a creative (as opposed to productivity or analysis) context.

\section{Task Analysis of DMI Mapping}
\begin{comment}
Task analysis and human-computer interaction: approaches, techniques, and levels of analysis - Abe Crystal, Beth Ellington
\end{comment}

The section on task analysis in Chapter 3 concluded that, as a technique, it is best-suited to understanding how to design interfaces that maximize productivity as opposed to channeling creative expression. This may or may not present a problem depending on whether we choose the characterize the use of computers, in the context of performing the task of mapping, as an act of expression or an act of productivity. Atau Tanaka offers a line of thinking that applies to tools and instruments, but may be of use in attempting to resolve this problem \cite{tanaka2000}.

\begin{quote}
``A tool can be improved to be more efficient, can take on new features to help in realizing its task, and can even take on other, new tasks not part of the original design specification. In the ideal case, a tool expands the limits of what it can do. It should be easy to use, and be accessible to [sic] wide range of naive users. Limitations or defaults are seen as aspects that can be improved upon.

A musical instrument's raison-d'etre, on the other hand, is not at all utilitarian. It is not meant to carry out a single defined task as a tool is. Instead, a musical instrument often changes context, withstanding changes of musical style played on it while maintaining its identity. A tool gets better as it attains perfection in realizing its tasks. The evolution of an instrument is less driven by practical concerns, and is motivated instead by the quality of sound the instrument produces. In this regard, it is not so necessary for an instrument to be perfect as much as it is important for it to display distinguishing characteristics, or "personality". What might be considered imperfections or limitations from the perspective of tool design often contribute to a "personality" of a musical instrument.

Computers are generalist machines with which tools are programmed. By itself, a computer is a tabula rasa, full of potential, but without specific inherent orientation. Software applications endow the computer with specific capabilities. It is with such a machine that we seek to create instruments with which we can establish a profound musical rapport.

The input device is the gateway through which the user accesses the computer software's functionality. As a generalist device, generalized input devices like the keyboard or mouse allow the manipulation of a variety of different software tools. Music software can be written to give musically specific capabilities to the computer. Input devices can be built to exploit the specific capabilities of this software. On this general platform, then, we begin to build a specialized system, each component becoming part of the total instrument description." 
\end{quote}

This line of thinking suggests that the extent to which Vizmapper is not a tool limits the extent to which the principles of user interface and data visualization design ought to play a role in the design process. HCI is much better suited to evaluating more objective notions like utility, tasks, and functionality than more subjective notions like personality, quality of sound, and musical rapport.

It is clear from Tanaka's definitions that Vizmapper serves as a tool rather than an instrument. However, the fact that Vizmapper is specifically a tool for accomplishing the task of creating and modifying mappings \emph{within a musical instrument} suggests that the design must be treated more subtly in this particular context. As the purpose of Vizmapper is partly to make the connections between components in a musical instrument more malleable and more susceptible to experimentation for groups of non-programmers, the task bears some resemblance to a non-utilitarian task. It is reasonable to assume that if the evolution of an instrument is motivated by the quality of the sound that the instrument produces, then similarly the evolution of a mapping interface is motivated by the quality of the mappings and DMIs that the interface produces. This reality should be evaluated in parallel with the understanding that configuration of a mapping is a somewhat utilitarian task; either the interface allows a team to configure the mapping efficiently or it does not.

To see this, the complex task of designing a musical instrument is deconstructed using an analysis process called \emph{hierarchical task analysis} \cite{annett1967}. 

Wanderley and Depalle distill the last decade of thinking into understanding the abstract components of digital musical instruments by decomposing a DMI into 3 main components \cite{wanderley2004}:

\begin{description}
\item \emph{The physical interface} containing the sensors, actuators, and physical body of the instrument.
\item \emph{The software synthesis system} which creates both the sonic output of the instrument and any visual, haptic and/or vibrotactile feedback.
\item \emph{The mapping system} in which connections are made between parameters of the physical interface and those of the synthesis system.
\end{description}

The design of these 3 components can be regarded as the 3 main subtasks of the overall task of designing a DMI. Each of these subtasks are themselves composed of more granular tasks, thus forming the beginnings of a hierarchical structure representing an analysis of the overall task.

\begin{description}
\item \emph{Designing the physical interface}
\begin{description}
\item choose sensors that are capable of detecting the desired physical phenomena or gestures
\item choose actuators that are capable of inducing the desired physical phenomena or feedback
\item choose sensors that are made available as outputs for connections to the inputs of the synthesis system
\item choose actuators that are made available as inputs for connections from the outputs of the synthesis system
\item choose structural/aesthetic materials that combined with the sensors and actuators will form the composite form of the physical interface
\item design the shape, look, feel of the composite physical interface
\end{description}
\item \emph{Designing the software synthesis system}
\begin{description}
\item choose a mechanism/algorithm for performing sound synthesis in software
\item choose a mechanism/algorithm for generating visual, haptic, and/or vibrotactile feedback
\item choose parameters of sound synthesis that are made available as inputs for connections from the outputs of the physical interface and/or other components of the composite synthesis system
\item choose parameters of visual, haptic, and/or vibrotactile feedback synthesis that are made available as inputs for connections from the outputs of the physical interface and/or other components of the composite synthesis system
\item choose parameters of sound, visual, haptic, and/or vibrotactile synthesis that are made available as outputs for connections to the physical interface and/or other components of the synthesis system
\end{description}
\item \emph{Designing the mapping}
\begin{description}
	\item choose a subset from the complete set of available outputs to start connections from
	\item choose a subset from the complete set of available inputs to connect to specific outputs
	\item create functional transformation mappings that specify how (if at all) to modify source signals from outputs to generate the desired destination signals for the inputs
\end{description}
\end{description}

This hierarchical model of DMI design generalizes well to anything from a self-contained handheld instrument to a massively distributed system spread over a large physical environment. It also works regardless of whether it is an individual or a team that is engaged in the instrument design. This decomposition also makes more clear why perhaps the mapping system is deserving of a dedicated system and user interface to manipulate. 

A good analogy is the choice of what collection of LEGO blocks to use as opposed to how to put the LEGO blocks together once the blocks are chosen. Procedurally, one chooses what types of blocks to use before deciding how to put the blocks together. Similarly, it makes sense to make certain choices about what pieces will be used for the physical interface and the software synthesis system before deciding how the mapping system will put the two collections of inputs/outputs together. It is difficult to map between two sets of signals when the sets of signals are not already defined. Of course, this model laying out the necessary choices to be made says nothing about \emph{what} choices should be made. That process is entirely dependent on the artistic intentions of the design team and the goal of a mapping tool is only to allow the choices that are made in the mapping subtask to be implemented as efficiently as possible.

\section{Implementation}

As a piece of software, there are a limited number of options for implementing the system while ensuring that the system can be reliably used on the variety of systems that a team designing a musical system is likely to be using in the present and future. As of the time that this system has been implemented, the most common form factors for computers are desktops, laptops, tablets, and smartphones. Each form factor is typically loaded with one of a small collection of operating systems that are widely available and easy to find expert knowledge about.

\begin{description}
\item[Desktops/Laptops] Windows (XP, Vista, 7), Mac OSX, Linux (Ubuntu, Debian, etc.)
\item[Tablets/Smartphones] Android, iOS
\end{description}

This is not an exhaustive list, however the point is that it is not practical (especially in a research context) to develop and maintain several versions of a software system that will work on a wide variety of operating systems and devices. However, the fact is that people will use whatever system they have available and the variety of devices that an artist or engineer is likely to use to interact with a sensor network is increasing. To design a practical system that is meant to work in a collaborative context, it is crucial to adapt to this reality.

Before beginning development on a software system, it is necessary to choose a development environment and a deployment environment. There are any number of very stable \emph{development} environments that enable a software developer to write a single collection of programmer code and translate the code into executables that can run on Windows, OSX, and Linux machines. These choices effect the developer of the software and not the user of the software. The choice of \emph{deployment} environment is much more consequential to the user because it effects how they will actually run the executable. For an explanation of technical terms pertaining to this chapter review the appendix.

There are three common methods for a user to run software on their chosen machine.

\begin{enumerate}
\item The first method is for the user to run a compiled machine code executable, which means that the developer has written code and translated the code with a compiler into machine code that can be executed without any further translation on the user's specific machine and operating system. If there is no available compiled executable for a specific machine architecture, the user can obtain the programmer code and compile it into an executable that works on their specific machine, as long as a compiler exists for their machine. This is how many video games are deployed.

\item The first method is an option when the software is developed in a \emph{compiled language} environment. There are also environments that use \emph{interpreted languages} or \emph{script languages}. An interpreted language differs from a compiled language because unlike a compiled language that uses the process of translating the program code into machine code with a compiler and producing a machine code executable to run the program, an interpreted language uses the process of interpreting the program code line by line directly when the program is run. The interpreter is responsible for interfacing with the machine for the interpreted code when the executable is run. In a interpreted language the programmer code and the executable are often the same file. Conversely, a compiled executable interfaces directly with the machine because it has already been converted into machine code. 

A web application is essentially an interpreted executable because a web browser that runs HTML, CSS, and Javascript downloads the programmer code and uses an internal interpreter to execute the program. It does not download a compiled executable. Distributing software as an interpreted script executable to be run in an interpreter is attractive because the burden for implementing the interpreter for various different machine architectures is moved to the developer of the programming language and the developer can count on a user being able to run their program as long as the interpreter for the language is available for the user's machine. The developer can write \emph{machine-independent} code.

\item Environments like Java use a third strategy and run software on \emph{virtual machines}. A virtual machine language is somewhat like a hybrid of a compiled language and interpreted language because a virtual machine language is compiled, but it is compiled to an intermediate machine code that is not the machine code of the actual physical machine that is running the program. It is compiled to the ``machine code" of a software virtual machine that must be implemented for all machines that one would like to run the program on. This is useful because the user can be given a single file that is not programmer code like a script executable, but the developer can still count on the file being able to run on any machine that has the virtual machine available on the system.
\end{enumerate}

Assuming the intent is for multiple team members to be running the Vizmapper interface to enable simultaneous work on the mapping of a network and that there is presumably a local network available to allow the various devices on the sensor network to communicate with each other, suggests Vizmapper ought to be implemented as a web application. 

A web application is typically distributed as a two part software system where one part is called the \emph{client} and the other part is called the \emph{server}. One server can serve the web application to a large number of clients. This is the source of another advantage of web applications. A web client runs in a web browser and therefore can be run on any machine that includes a modern web browser, which includes every operating system listed at the beginning of this section. All modern web browsers provide an increasingly identical deployment environment. Since Libmapper is compiled code, the server software can take care of interfacing with the library and communicating through the Mapper protocol with the rest of the Mapper network. Despite the fact that the server is more restricted, in terms of what deployment environment it can be run on because of the Libmapper dependency, the server need only run on one machine. Also, a web application is much easier to deploy on a collection of heterogenous machines. At this point in time, a web application reduces the likelihood that the team will be forced to huddle around the chosen few machines that can run the client, even if all machines cannot run a server with a more restrictive deployment environment.

Stephen Sinclair, the developer of Libmapper and a member of IDMIL, also developed a basic server written in Python (a popular interpreted language) that interfaces with Libmapper and provides a convenient interface for a web browser client to indirectly speak the Mapper protocol to the local network of devices by communicating with the server. This web server is called Webmapper. Together, Webmapper/Vizmapper provides a server/client web application that can be used to configure a Mapper network.

\section{Application of User Interface Design Principles}
\label{sec:userInterface}

If the user of Vizmapper is a human user configuring an MSN, then Vizmapper is the ``user" of Webmapper. Webmapper is then the ``user" of Libmapper. Libmapper provides an interface for Webmapper to speak the Mapper protocol and Webmapper provides an interface for Vizmapper to indirectly speak the Mapper protocol. Now the task remains to decide how the Vizmapper interface will present the signals and devices present on the network to the human user and allow the user to configure mappings between these signals and devices through Webmapper.

As a reminder, the subtask of \emph{Designing the mapping} was further broken down as follows:

\begin{description}
	\item choose a collection of available outputs to start connections from
	\item choose a collection of available inputs to connect to specific outputs
	\item create mappings that specify how (if at all) to modify source signals from outputs to generate the desired destination signals for the inputs
\end{description}

From the hierarchical task analysis of designing a DMI, it can be argued that the task of designing a mapping is a less complex task than the task of designing a physical interface or designing an audio synthesis system. This lack of complexity in the task is one of the reasons why it is possible to even contemplate performing the other task through a single interface. Building an interface for the task of designing a physical interface or designing an audio synthesis system is less manageable in this way.

Instead, the complexity of the task lies in the complexity of the Mapper network and what the task of \emph{choosing} actually entails. Since the focus is now on the mapping task, it is helpful to see if it is possible to be even more granular in the hierarchical decomposition of the task. 

\begin{description}
	\item choose a collection of available outputs to start connections from
    \begin{description}
        \item view the complete set of output signals
        \item find an output signal to connect to an input signal
        \item select the output signal
    \end{description}
	\item choose a collection of available inputs to connect to specific outputs
    \begin{description}
        \item view the complete set of input signals
        \item find an input signal to connect to an output signal
        \item select the input signal
    \end{description}
	\item create mappings that specify how (if at all) to modify source signals from outputs to generate the desired destination signals for the inputs
    \begin{description}
        \item define a functional transformation/signal conditioning mapping between the output signal and input signal 
        \item connect the signals
    \end{description}
\end{description}

This clarifies the individual steps implicit in the overall task and suggests how to distribute the steps needed in the interface. Since the screen space on a monitor is limited and the most important data about the Mapper network are the names of the signals and the mappings between them, devoting as much space to the visualization of the signals and their connections is a priority.

The first decision made is to separate the viewing and browsing of the MSN from the editing of mappings. Vizmapper treats these two sets of subtasks as two different modes. Entering viewing and browsing mode means that it is not possible to modify the mappings of the signals being browsed. Entering editing mode, focuses the view on a subset of signals from the complete set and means that it is not possible to browse the larger set of signals. This decision is made because viewing and browsing does not actually effect the Mapper network, only editing a mapping effects the network. Preventing both modes from being accessed on the same screen makes the conceptual separation between the modes clear and reduces the likelihood of accidental modifications.

When it comes to viewing and browsing, since the interface is being designed with larger MSNs in mind, there should be a way to filter the complete set of signals to a subset of signals. As discussed in Chapter 3, this can be accomplished by pointing and clicking to utilize recognition processes or it can be accomplished by typing to encourage the use of recall processes. Although finding a specific signal through name recognition is less straining then remembering the name of a specific signal and typing it, it is not necessarily a simple decision. In the case where a user has been working with the same Mapper network for an extended period of time, it may become tedious to point, click, and browse their way to a specific signal even though they know the exact name of the signal they are looking for. In this case, the physical process involved to select the signal becomes more consciously cumbersome than the cognitive process involved in recognition or recall.

A single text input box, for entering a text matching filter query, occupies very little vertical space and adds functionality that is helpful for this type of user. The user can type in a portion of the complete OSC name of a signal and limit the view to only the signals that match the text in the input box. In order to display the union of more than one text query (show the signals that match the first text query OR second query OR third query, etc.), the user uses a space character to delimit the separate filter queries. 

Typically, interfaces that allow the user to browse data using text queries allow the user to compose more complex queries (like ANDs or NOTs in addition to ORs). However, this functionality seems to deviate from the scope of the interface and add syntax complexity without a substantial benefit to the central task. Since the user is only searching within a set of names of signals and not a set of large text documents (like one might do with a database or the internet), it is likely this power is not required by most users. Additional features and complexity are avoided in Vizmapper, unless the functionality that is added is significant to the mapping task as it has been analyzed.

\section{Application of Data Visualization Design Principles}

Following the graphical perception research in Section \ref{sec:Graphical Perception}, before making decisions about the visualization of the network, the data set of a Mapper network will be analyzed using the 3 stage process of Bertin \cite{semiology1983}.

This involves determining the \emph{components} in the data, the \emph{length} of each component, and the \emph{level of organization} (nominal, ordinal, or quantitative).

There are 3 easily distinguishable sets of data that a user concerns themselves with when mapping signals.

\begin{enumerate}
\item set of signals on the Mapper network, which are defined as OSC names that are part of a hierarchical namespace
\item set of associative connections between pairs of signals, each connection is defined by a source and a destination (the source of the data is typically an output signal and the destination is typically an input signal)
\item set of function transformations that are paired with each associative connection, which are defined using a function type, a mathematical function, and an allowable range of values for the input of the function (actually an output signal in Mapper terminology) and output of the function (actually an input signal in Mapper terminology)
\end{enumerate}

It is useful to note that the most complex elements are the elements of the set of functional transformations. The least complex elements are the OSC names of the signals. This is taken simply from the number of variables that must be defined to fully define each element.

So the data set can be decomposed into \emph{3} components, but it might also be decomposed into \emph{4}. The first component, the set of signals, can be split into one component for output signals and one component for input signals. The question is whether one decomposition is more effective than the other. If it is assumed that the source of a connection will always be an output signal of a device and the destination of a connection will always be an input signal of a device, it makes sense to use \emph{4} because it will always be necessary to select a source and a destination to form any connection and it will be efficient to graphically distinguish between the 2 types of signal. Therefore, there are \emph{4} components in the data set. The first component of the previous decomposition can be further split into:

\begin{enumerate}
\item set of output signals on the Mapper network, which are defined as OSC names that are part of a hierarchical namespace
\item set of input signals on the Mapper network, which are defined as OSC names that are part of a hierarchical namespace
\end{enumerate}

The length of the output signals component and the input signals component are assumed to be \emph{large} and within an order of magnitude of each other. There might be many more output signals than input signals or vice versa, however since there will be roughly one control mechanism device with output signals for each sound generation mechanism device with input signals in a scenario with DMIs, this is a reasonable assumption.

The length of the set of associative connections is bounded by the length of the set of input signals. This is because the Mapper protocol only handles \emph{one-to-many} and \emph{one-to-one} mappings as explained in Chapter 2. Therefore, there can be, at most, one connection for every input signal.

The length of the set of function transformations is equal to the length of the set of associative connections because every connection is paired with a functional transformation governing data transmission over the connection. This is not strictly true, however it is simpler to treat those connections with no active transformation as utilizing the function of \begin{math}x = y\end{math} with no bounded ranges (also known as a bypass) to indicate the lack of any signal conditioning.  

The level of organization of all four components is \emph{nominal}. None of the components have elements that can be ordered or manipulated with arithmetic.

So the conclusion of a Bertin-like analysis of a typical Mapper network data set is the following:
\begin{description}
\item \emph{Components} - four components
    \begin{enumerate}
        \item output signals
        \item input signals
        \item connections
        \item functional transformations
    \end{enumerate}
\item \emph{Length} - all four components are of approximately equivalent length
\item \emph{Level of organization} - all four components are nominal
\end{description}

This analysis comes with one caveat in light of Chapters 2 and 3. There is one level of organization that Bertin does not address - hierarchical organization. This is a level of organization that seems relevant, especially given the importance of hierarchical organization in human cognition, yet is not directly addressed by the nominal/ordinal/quantitative categorization scheme. Due to the nature of OSC namespaces, the set of output signals and the set of input signals can be grouped hierarchically. This suggests a type of relation between signals that is not simply = or NOT =.

The problem of visualizing an abstract data set that has both hierarchical relationships between elements and connections between elements as components is problem that has been tackled in data visualization literature \cite{edgebundles2006}. Holten refers to the hierarchically organized component as a set of \emph{inclusion relations} and the set of connections as a set of \emph{adjacency relations}. One example of a data set exhibiting a structure similar to a Mapper network data set is an academic citation network where publications are the lowest level elements of the hierarchy and departments and universities are represented by higher level groupings in the hierarchy. Connections between publications would represent one publication citing another \cite{edgebundles2006}.

Since we are dealing with nominal data with some hierarchical organization, let us focus on the most discerning perceptual tasks for nominal data using the ranking presented in Section \ref{sec:Graphical Perception}. This reduced set of ranked perceptual tasks is shown in Table \ref{tab:perceptualNominal}.

\begin{table}
    \begin{center}
    \begin{tabular}{ | l | }
    \hline
    Nominal \\ \hline
    Position \\
    Color Hue \\
    Texture \\
    Connection \\
    Containment \\
    Density \\
    \hline
    \end{tabular}
    \end{center}
    \caption{Ranking of top perceptual tasks for discerning between nominal data \cite{jock1986}}
    \label{tab:perceptualNominal}
\end{table}

Inspired by Holten's own exploration of visualizing similarly structured data, the decision is made to represent signals with circles occupying different points in space. The heuristic reason for this decision is that the shape of a circle lends itself to a symmetrical and aesthetically pleasing presentation of the data and makes for an easy target for the user to click (more on this shortly). There are 4 components and it is important to distinguish between these components, first and foremost. Since position is already being used to distinguish between signals and at least one output signal and one input signal will need to be displayed for the connection and functional transformation components to be relevant, it makes sense to place a constraint on the \emph{position} of signals that forces the output signals to be confined to the left side of the display and input signals to be confined to the right.

\begin{figure}[htb]
\centering
\includegraphics[width=0.6\textwidth]{holten_balloon.png}
\caption{Example of a balloon tree visualization with dark lines for inclusion relations and light lines for the adjacency relations - inspired by \cite{edgebundles2006}}
\label{fig:balloonTree}
\end{figure}

The decision is made to use \emph{containment} to represent the inclusion relations between hierarchically organized elements. This is taken from the use of a \emph{balloon tree} visualization by Holten in his own research. An example of a balloon tree is shown in Figure \ref{fig:balloonTree}. Examining the figure, notice that \emph{connection}, \emph{containment}, and \emph{density} are used to communicate specific relationships between elements. Again, there are two components being displayed: a set of inclusion relations and a set of adjacency relations. In this example, there are only two adjacency relations shown in the figure. 

Connection is used to communicate both adjacency and inclusion. The curved, lighter lines represent adjacency relations between terminal elements (for example, citations between publications). The straight, dark lines represent inclusion relations between different levels in the hierarchy. Density and shape are used to distinguish between the two. Density is also used to distinguish between terminal elements in the hierarchy (dark dots) and intermediate elements in the hierarchy (light dots, for example, representing departments and universities). 

Containment and position are also used to communicate relations between the elements. Enclosing circles differentiate between levels in the hierarchy and different branches of the hierarchy on the same level. The center of every circle is the position of the intermediate elements defining what the circle represents (for example, department on one level of hierarchy, university on another level) and the elements positioned radially around the centers are the child elements (either terminal or lower level, but still intermediate, elements). Notice that terminal elements can either be placed directly on the first level of the hierarchy or on deeper levels depending on the specific inclusion relations of the data set (perhaps there are independent publications that are not associated with any particular organization in the data set).

\begin{figure}[htb]
\centering
\includegraphics[width=0.55\textwidth]{vijay_balloon.png}
\caption{Modified balloon tree visualization}
\label{fig:balloon}
\end{figure}

The balloon tree used in Figure \ref{fig:balloonTree} is not as uncluttered as it might be. The lines used to communicate inclusion relations and the lighter dots placed at the center of every circle actually display the same data as the containing circles. This seems to be unnecessary visual redundancy. Figure \ref{fig:balloon} effectively communicates the same data set as \ref{fig:balloonTree} with no loss of information and less clutter. The missing visual elements are implied.

\begin{figure}[htb]
\centering
\includegraphics[width=0.55\textwidth]{clutter_balloon.png}
\caption{Cluttered balloon tree visualization}
\label{fig:clutterBalloon}
\end{figure}

When the number of elements in the data set grows large and there are many adjacency relations between the elements, there is a different type of clutter that arises involving the lines representing the adjacency relations like in Figure \ref{fig:clutterBalloon}. This is a type of clutter that is more difficult to reduce. One can be more specific about how the connections are drawn, using bezier curves and bundling to organize the connections like one might bundle wires between mixers. This is the method that Holten uses \cite{edgebundles2006}. In future work, using Holten's connection clustering algorithm in Vizmapper would be a net benefit, however it is not included at the time of this thesis. 

As an alternative to Holten's method, it may be possible to use a different graphical property entirely. However, it is difficult to conceive of a graphical property that communicates connections between signals better than the graphical property of connection. One potential way out is realizing that there is no need to treat the visualization as static. It is already possible to use a text-matching filter to modify the visualization by typing within the interface as described in Section \ref{sec:userInterface}. Perhaps, it would make sense to allow the user to filter the visualization by pointing and clicking as well.

\begin{figure}[htb]
\centering
\subfloat[Top level][Top level]{
\includegraphics[width=0.33\textwidth]{firstLevelBalloon.png}
\label{fig:unclutteredBalloon}}
\subfloat[Branch of the second level][Branch of the second level]{
\includegraphics[width=0.33\textwidth]{secondLevelBalloon.png}
\label{fig:secondUnclutteredBalloon}}
\subfloat[Branch on the third level][Branch on the third level]{
\includegraphics[width=0.33\textwidth]{thirdLevelBalloon.png}
\label{fig:thirdUnclutteredBalloon}}
\caption{Interactive balloon tree visualization of hierarchy}
\label{fig:interactiveBalloon}
\end{figure}

Figure \ref{fig:unclutteredBalloon} only displays details of the hierarchy at the top level of the current view. It is clear that this alternative bubble tree leaves out information present in the other visualizations, even though it is supposed to visualize the same data set as Figure \ref{fig:clutterBalloon}. However, it displays all the adjacency relations between groups on single level of the hierarchy and displays the inclusion relation between the root of the hierarchy and the first level of the hierarchy. If the user clicks on gray circle in Figure \ref{fig:unclutteredBalloon} representing a branch of the hierarchy on the second level, then the interface can modify the visualization to Figure \ref{fig:secondUnclutteredBalloon}. The user can then click on the gray circle in Figure \ref{fig:secondUnclutteredBalloon} to proceed into the deepest level (which in Figure \ref{fig:clutterBalloon} is the third level) in the whole hierarchy shown in Figure \ref{fig:thirdUnclutteredBalloon}. 

However, this interactive visualization has lost information about the adjacency relations between the current level of the hierarchy and a higher level of the hierarchy. It only displays adjacency relations between the groups on the same level. Therefore, even though Figure \ref{fig:unclutteredBalloon} shows the gray circle having adjacency relations between all the other groups and elements on the top level of the hierarchy, Figure \ref{fig:secondUnclutteredBalloon} loses this information completely and there is no way to recover exactly what terminal elements are connected to each other.

\begin{figure}[p]
\centering
\includegraphics[width=0.88\textwidth]{maxmapperFirst.png}
\caption{Example Mapper network displayed in Maxmapper - Meta-Instrument and Granul8}
\label{fig:maxOne}
\end{figure}

\begin{figure}[p]
\centering
\includegraphics[width=0.88\textwidth]{maxmapperThree.png}
\caption{Example Mapper network displayed in Maxmapper - Minibees and Modal}
\label{fig:maxThree}
\end{figure}

Fortunately, a decision made in Section \ref{sec:userInterface} allows this problem to be circumvented. The problem of lost information in this scenario is a result of all of the elements indiscriminately belonging to the same component. Since the decision is made in Vizmapper to separate the potential sources of a connection from the potential destinations of a connection and any given signal can only be an output (source) or input (destination), this problem is nonexistent.

\section{Vizmapper}

The best way to see this is to see the visualization used in Vizmapper. The example Mapper network that will be used in the remainder of this chapter features Serge de Laubier's Meta-Instrument \cite{metainstrument1998} and a collection of four wireless Minibee transmitters, each with a 3-axis accelerometer, as the control devices. The synthesizer devices are the Granul8 granular synthesizer and a modal synthesizer, both developed at IDMIL and implemented using Max/MSP. 

\begin{figure}[p]
\centering
\includegraphics[width=0.88\textwidth]{vizmapperFirst.png}
\caption{Top level of the Mapper network signal hierarchy with the Meta-Instrument highlighted}
\label{fig:vizOne}
\end{figure}

\begin{figure}[p]
\centering
\includegraphics[width=0.88\textwidth]{vizmapperSecond.png}
\caption{Top level of the Meta-Instrument with the \textbf{raw} signal cluster highlighted}
\label{fig:vizTwo}
\end{figure}

\begin{figure}[p]
\centering
\includegraphics[width=0.88\textwidth]{vizmapperThird.png}
\caption{\textbf{/meta.1/raw} branch with \textbf{left} signal cluster highlighted}
\label{fig:vizThree}
\end{figure}

\begin{figure}[p]
\centering
\includegraphics[width=0.88\textwidth]{vizmapperFour.png}
\caption{\textbf{/meta.1/raw/left} branch with \textbf{ring} finger cluster highlighted}
\label{fig:vizFour}
\end{figure}

\begin{figure}[p]
\centering
\includegraphics[width=0.88\textwidth]{vizmapperFive.png}
\caption{\textbf{/meta.1/raw/left/ring} branch with 5 signals in this cluster displayed}
\label{fig:vizFive}
\end{figure}

\begin{figure}[p]
\centering
\includegraphics[width=0.88\textwidth]{vizmapperSix.png}
\caption{\textbf{/minibee.1/node.3/accel} branch of output signals and \textbf{/granul8.1/grain.1} branch of input signals}
\label{fig:vizSix}
\end{figure}

Figure \ref{fig:maxOne} shows Maxmapper displaying the Meta-Instrument output signals in the \emph{Source} column and the Granul8 input signals in the \emph{Destination} column. As shown at the bottom of the interface, the Meta-Instrument declares 101 output signals and the Granul8 synthesizer declares 191 input signals. Figure \ref{fig:maxThree} shows the 4 Minibees with a total of 12 output signals (3 signals each) and the modal synthesizer with 39 signals. The idea is that Vizmapper presents a different view of this large system, which makes more clear the relationships between signals.

Figures \ref{fig:vizOne} through \ref{fig:vizFive} show screenshots of a Figure \ref{fig:interactiveBalloon}-style branch traversal down the Meta-Instrument in Vizmapper. Every highlighted cluster in these figures indicate a user click on the cluster to transition to the next screen in the interface. This series of screenshots shows the implementation of the interactive, balloon tree visualization arrived at in Figure \ref{fig:interactiveBalloon}. 

The inclusion relations between all signals on the Mapper network are inferred by Vizmapper from the OSC namespace of the Mapper network. This places a premium on a wisely constructed OSC namespace for the signals that reduces the number of signals at any given level of a branch of the hierarchy. The OSC namespace of the Mapper-enabled Meta-Instrument is an example of a well-constructed namespace. As explained in Chapter 3, this type of visual hierarchical organization reduces the number of elements that a user has to decide between on any given level of a branch of the hierarchy and should mean a more effective interface. Notice the use of whether the circle is filled or not to represent whether the circle represents a terminal signal or a cluster of signals at an intermediate level of one branch of the complete hierarchy. \emph{Cluster} and \emph{branch} will be used interchangeably to refer to a specific position in the network hierarchy. Additionally, the Vizmapper visualization of the network displays two levels of the hierarchy at any given point, giving the user more context then the one level view used in Figure \ref{fig:interactiveBalloon}.

The center of the screen displays two columns of buttons that display a trace of the path taken in the OSC namespace hierarchy for the output signals and for the input signals. Clicking on one of the buttons in a column returns the visualization of either the outputs signals or input signals to a previous position in the current traversal. Clicking on the always present \emph{output signals} and \emph{input signals} buttons returns the visualization of that set of signals to the first level of the hierarchy. The left and right columns of text display the names of the currently accessible groups and/or signals at the current level of the hierarchy. The hierarchical visualization and text-matching filter at the top of the screen work together by removing any groups or signals from the visualization whose paths do not match the text filter.

In order to make a connection, a user uses the \emph{view mode} to navigate down to a level of both the output signal hierarchy and input signal hierarchy that have at least one terminal signal in each. This is because the current view of the network is frozen when the user switches to the \emph{edit mode} by clicking the edit tab at the top left of the screen. Hierarchy traversal is limited to the view mode. Figure \ref{fig:vizSix} shows an example traversal in view mode to prepare for connections between output signals in the \verb#/minibee.1/node.3/accel# cluster and input signals in the \verb#/granul8.1/grain.1# cluster.

\begin{figure}[htb]
\centering
\includegraphics[width=0.88\textwidth]{vizmapperSeven.png}
\caption{Creation of new connection between \textbf{/minibee.1/node.3/accel/x} signal and \textbf{/granul8.1/grain.1/beginratio} signal in edit mode}
\label{fig:vizSeven}
\end{figure}

\begin{figure}[hp]
\centering
\includegraphics[width=0.88\textwidth]{vizmapperEight.png}
\caption{Creation between two more pairs of signals in same branch as first connection}
\label{fig:vizEight}
\end{figure}

\begin{figure}[hp]
\centering
\includegraphics[width=0.88\textwidth]{vizmapperNine.png}
\caption{After more connections between the other 3 Minibees and \textbf{/minibee.1/node.7}, \textbf{/minibee.1/node.8}, \textbf{/minibee.1/node.9} branches}
\label{fig:vizNine}
\end{figure}

Once the two signals that are to be connected are in view, the user switches to the \emph{edit mode}, clicks on one output signal and one input signal, enters the pertinent information into the form on the left side of the screen to define a functional transformation (if necessary), and then clicks the \emph{update} button to tell the Webmapper server to request a connection between the two selected signals. Any device linking that is necessary to create the needed Libmapper routers to manage the connection is done automatically without user intervention.

As of this thesis, Libmapper allows network monitors to specify a functional transformation with the several properties. Vizmapper provides an interface to a subset of these properties to simplify the choices of the user:

\begin{description}
\item{Mode} - There are 5 modes that can be selected.
\begin{description}
\item{None} - This acts as a default in Vizmapper. If the min and max range of the output and input signals are defined in the signal declaration then Libmapper will request a line transformation with the expression of the line chosen to cover both ranges linearly. Otherwise, it will request a bypass transformation.
\item{Line} - This mode applies a linear transformation defined by the specified expression, source range, and destination range.
\item{Bypass} - This mode represents a lack of functional transformation and applies no conditioning to the output signal before sending the data to the input signal.
\item{Expression} - This mode accepts an arbitrary single-variable expression to condition the signal.
\item{Calibrate} - If the input signal has a defined range, this mode monitors the data over the connection until a line or expression mode request is sent and constructs a transformation calibrated to the data transmitted over the connection during this time. If the destination signal has no defined range it defaults to a bypass.
\end{description}
\item{Expression} - The expression used in a line or expression transformation.
\item{Source range} - The range that the output signal data is restricted to.
\item{Destination range} - The range that the input signal data is restricted to.
\end{description}

The user can view the functional transformation properties of a connection at any time by clicking on the curved line that represents the connection while in edit mode. Clicking on a connection while in view mode does nothing.

To break a connection between two signals or view the active functional transformation over the a connection, the user clicks on a connection between two signals to select the connection. To finish breaking the connection, the user clicks the \emph{remove} button.

Figure \ref{fig:vizSeven} shows the result of creating a default \emph{None} connection between the \verb#x# output signal in the \verb#/minibee.1/node.3/accel# cluster and the \verb#beginratio# input signal in the \verb#/granul8.1/grain.1# cluster. The mode, expression, source range, and destination range have been automatically calculated by Libmapper to fit the ranges of the signals that were connected. After making two more connections between the \verb#y# and \verb#length# signals, and the \verb#z# and \verb#gain# signals, the interface looks like Figure \ref{fig:vizEight}.

The same 3 signal connections are made for each of 3 more pairs of clusters: \verb#/minibee.1/node.7# and \verb#/granul8.1/grain.2#, \verb#/minibee.1/node.8# and \verb#/granul8.1/grain.3#, \verb#/minibee.1/node.9# and \verb#/granul8.1/grain.4#. If the user navigates to the top level of \verb#/minibee.1# and \verb#/granul8.1# in view mode, the resulting display looks like Figure \ref{fig:vizNine}.

\begin{figure}[hp]
\centering
\includegraphics[width=0.88\textwidth]{vizmapperTen.png}
\caption{Top level view of Mapper network signal hierarchy with all new connections summarized as 4 visual connections}
\label{fig:vizTen}
\end{figure}

\begin{figure}[hp]
\centering
\includegraphics[width=0.88\textwidth]{vizmapperEleven.png}
\caption{Use of signal filter in top view to focus on relevant signal clusters}
\label{fig:vizEleven}
\end{figure}

Anecdotally, after working with this particular Mapper network for some time, the visual patterns representing clusters of signals begin to be recognizable and it is possible to navigate through the network without paying too much attention to the textual labels of clusters and signals while in view mode. This is not the case when two clusters contain a similar arrangement of subclusters and signals, which is one shortcoming of this visualization. In future work, it may be fruitful to explore the idea of automatically generating a distinct visual pattern for each cluster and signal that guarantees elements in the same branch of the hierarchy are readily distinguishable from each other.

At the top level of the whole Mapper network, the amount of visual clutter is reduced because the 12 connections that were created are reduced to 4 visual connections (one curved line per Minibee node/Granul8 synth grain pair) as shown in Figure \ref{fig:vizTen}. Figure \ref{fig:vizEleven} shows the result of reducing the visual clutter further by using the signal filter functionality of the text input box at the top of the interface. This filtering mechanism can be used at any point in either the view or edit mode.

\begin{figure}[hp]
\centering
\includegraphics[width=0.88\textwidth]{maxmapperSecond.png}
\caption{Same connections shown in Maxmapper}
\label{fig:maxTwo}
\end{figure}

Comparing these screenshots of the Vizmapper with the Maxmapper display of the same Mapper network as shown in Figure \ref{fig:maxTwo}, there is a clear difference in the structural knowledge of the network that each interface provides. A benefit of the Libmapper system is that it allows multiple monitors using different GUIs to operate on the same Mapper network at the same time. Therefore, different team members can use different interfaces depending on their personal preferences. 

\section{Summary}

The Vizmapper interface provides all the functionality necessary to create a mapping on a Mapper network. The central argument is that through the process of understanding the mapping task and researching user interface design principles, data visualization principles, and human cognitive processes, this thesis research has resulted in an effective interface for configuring mappings that scales to a MSN with many signals.


%==========
\typeout{}
\resetdatestamp

\chapter{Conclusions and Further Research}

This concludes the research presented in this thesis. The base impulse behind this whole endeavour is the belief that the interface of a system is as important, arguably more important, then the internals of the system. The internals of a system are tasked with maintaining the stability, reliability, and functionality of the system, but the interface of the system to the broader set of external systems, users, and world is ultimately what gives a system purpose. The history of computer science and electrical engineering is largely a history of ignoring the interface or relegating the interface to a mere curiosity, simply the interest of the types of individuals who are too ``artsy" to appreciate ``real" computer science and programming languages. This is true of not only graphical user interfaces, but the interfaces of server-based application programming interfaces, code libraries, object oriented classes, and even the documentation (I admit to this shortcoming) that allows the external world to understand the ``protocol" of the interface in plain, human language. The distinction between the \emph{system} and the \emph{interface} is, at its heart, an artifact of the history of computers and the specific individuals who devoted their lives to the endeavour.

Noticeably absent in this thesis is any mention of user testing or experimentation to see whether the Vizmapper interface actually accomplishes what is intended. Such testing would certainly be expected of a larger project on this topic and would be helpful to refine the interface in future iterations. However, to steal a phrase, I greet the conventional understanding of the utility of user testing with skepticism. Sometimes an interface should be molded to the user and other times the user should be molded by the interface. That is the nature of compelling non-computational interfaces; it stands to reason that the same would be true of computational interfaces. The type of long term study that would be necessary to observe how interfaces affect the mindset of a user is not something that is often (if at all) performed. However, I think that such a study would be far more compelling and useful than the typical user testing that is performed.

There are two concrete improvements that should be made to the current Vizmapper interface in the future. The first is to use Holten's connection bundling algorithm \cite{edgebundles2006} to reduce the visual clutter when many connections are displayed simultaneously. The second is to use some form of visual hashing to automatically generate a distinct visual pattern for each cluster and signal so that visually identical signals and clusters can be readily distinguished between, as one grows familiar with their particular Mapper network. There is much work to be done to further improve interfaces to Libmapper if the Mapper Tools project is ever to be used in a wide context.


%========== Appendices
\appendix
\include{resources}
\include{definitions}

%==========
\begin{comment}
\typeout{}
\input{A-A}
\end{comment}

%========== Bibliography
\typeout{}
\begin{singlespace}
  \bibliography{vijay_thesis}
  \bibliographystyle{ieeetr}
\end{singlespace}

\end{document}
